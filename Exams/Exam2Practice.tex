%%%%%%%%%%%%%%%%%%%%%%%%%%%%%%%%%%%%%%%%%%%%%%%%%%%%%%%%%%%%%%%%%%%%%%%%%%%%%%%%%%%%%%%%%%%%%%%%%%%%%%%%%%%%%%%%%%%%%%%%%%%%%%%%%%%%%%%%%%%%%%%%%%%%%%%%%%%%%%%%%%%
% Written By Michael Brodskiy
% Class: Modern Physics
% Professor: Q. Yan
%%%%%%%%%%%%%%%%%%%%%%%%%%%%%%%%%%%%%%%%%%%%%%%%%%%%%%%%%%%%%%%%%%%%%%%%%%%%%%%%%%%%%%%%%%%%%%%%%%%%%%%%%%%%%%%%%%%%%%%%%%%%%%%%%%%%%%%%%%%%%%%%%%%%%%%%%%%%%%%%%%%

\include{Includes.tex}

\usepackage[dvipsnames,table]{xcolor}
\usepackage{siunitx} % SI-units
\usepackage{pgfplots}
\usepgfplotslibrary{units} % to add units easily to axis
\usepgfplotslibrary{fillbetween} % to fill inbetween curves
\usepgfplotslibrary{colormaps} % to create colormaps

\title{Exam 2 Practice Problems}
\date{March 29, 2023}
\author{Michael Brodskiy\\ \small Professor: Q. Yan}

\begin{document}

\maketitle

\newpage

\begin{enumerate}

    \section{Conceptual Questions}

  \item Infinite Wells and de Broglie Waves

    Because the wavelength may be defined as $\lambda=\dfrac{2L}{n}$, the situation is similar to the oscillation of a standing wave, like a string, fixed at two ends. As such, a particle moving in a similar manner would have standing de Broglie waves as solutions to the Schr\"odinger equation.

  \item Wave Normalization

    Purely mathematically, an un-normalized wave may be a solution to the Schr\"odinger equation; however, when applying this concept to physical quantities, it is necessary for it to be normalized. This is because the integral over the entire probability function must be equal to 1 or 100\%, and, thus, the function needs to be normalized to meet this requirement.

  \item The Physical Meaning of $\displaystyle \int_{-\infty}^{\infty} |\psi(x)|^2\,dx=1$

    $\psi(x)$ represents the wave function, which describes the probability of a particle being in a given position. Taking the magnitude of the wave function and squaring it, or $|\psi(x)|^2$ generates a probability density distribution. By integrating over the whole function, a probability of 1, or 100\% would be calculated, as it is definite that the particle is somewhere within the entirety of the wave.

  \item Harmonic Oscillators

    The ground state energy occurs when $n=0$; given the formula $E_n=\left(n+\frac{1}{2}\right)\hbar\omega_0$, this means the ground state energy is $E_0=\frac{1}{2}\hbar\omega_0$. The difference between an energy level and the ground state energy may be expressed as $E_n-E_0=n\hbar\omega_0$. This means that the smallest energy difference would be when $n=1$, and the second smallest would occur when $n=2$, yielding a value of $E_2-E_0=2\hbar\omega_0$

\end{enumerate}

    \section{Problems}

\begin{enumerate}

  \item Using Normalization Conditions

    Setting up the normalization equation, we get:

    $$\int_{-\frac{L}{2}}^0 \left(C\left(\dfrac{2x}{L}+1\right)\right)^2\,dx+\int_0^{\frac{L}{2}} \left(C\left(\dfrac{-2x}{L}+1\right)\right)^2\,dx=1$$

    To simplify integration we can do the following:

    $$\int_{-\frac{L}{2}}^0 \left(\dfrac{2xC}{L}+C\right)^2\,dx+\int_0^{\frac{L}{2}} \left(\dfrac{-2xC}{L}+C\right)^2\,dx$$
  $$C^2\int_{-\frac{L}{2}}^0 \left(\dfrac{4x^2}{L^2}+\dfrac{4x}{L} + 1\right)\,dx+C^2\int_0^{\frac{L}{2}} \left(\dfrac{4x^2}{L^2}-\dfrac{4x}{L} + 1\right)\,dx$$
  $$C^2\int_{-\frac{L}{2}}^{\frac{L}{2}} \left(\dfrac{8x^2}{L^2} + 2\right)\,dx$$

  Finally, we must solve:

  $$C^2\left( \dfrac{8x^3}{3L^2}+2x \right)\Big|_{-\frac{L}{2}}^{\frac{L}{2}}$$
  $$C^2\left( \left(\dfrac{L}{3}+L\right)-\left( -\dfrac{L}{3}-L \right) \right)$$

  Returning the ``$=1$'', we get:

  $$C^2\left( \dfrac{L}{3} \right)=1$$
  $$C^2=\dfrac{3}{L}$$

  Finally, we get:

  $$\boxed{C=\pm\sqrt{\frac{3}{L}}}$$

  \item Applying Boundary Conditions to Find Constants

    First and foremost, because $e^x$ diverges, we know that $\boxed{C=0}$. This leaves $A$, $B$, and $D$. We know the functions must be continuous at boundary, meaning that, at $x=0$, the functions need to equal each other:

    $$A\sin(k_0(0))+B\cos(k_0(0))=De^{-k_1(0)}$$
    $$B=D$$

    Next, we know that the first order derivatives of the function must be continuous as well. The derivatives are:

    $$\left\{\begin{array}{l r} Ak_0\cos(k_0x)-Bk_0\sin(k_0x), & x<0\\-Dk_1e^{-k_1x}, & x>0 \end{array}$$

    Similarly to the first step, we must plug in $x=0$, or the boundary where this changes at. This generates:

    $$Ak_0=-Dk_1$$

    Rearranging in terms of $A$, we get:

    $$\boxed{B=D=-\dfrac{k_0A}{k_1}}$$

\end{enumerate}

\end{document}

