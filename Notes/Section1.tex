%%%%%%%%%%%%%%%%%%%%%%%%%%%%%%%%%%%%%%%%%%%%%%%%%%%%%%%%%%%%%%%%%%%%%%%%%%%%%%%%%%%%%%%%%%%%%%%%%%%%%%%%%%%%%%%%%%%%%%%%%%%%%%%%%%%%%%%%%%%%%%%%%%%%%%%%%%%%%%%%%%%
% Written By Michael Brodskiy
% Class: Modern Physics
% Professor: Q. Yan
%%%%%%%%%%%%%%%%%%%%%%%%%%%%%%%%%%%%%%%%%%%%%%%%%%%%%%%%%%%%%%%%%%%%%%%%%%%%%%%%%%%%%%%%%%%%%%%%%%%%%%%%%%%%%%%%%%%%%%%%%%%%%%%%%%%%%%%%%%%%%%%%%%%%%%%%%%%%%%%%%%%

\include{Includes.tex}

\title{Introduction to Modern Physics}
\date{\today}
\author{Michael Brodskiy\\ \small Professor: Q. Yan}

\begin{document}

\maketitle

\newpage

\tableofcontents

\newpage

\begin{itemize}

    \section{Modern Physics}

  \item Modern physics is a set of developments that emerged around 1900

  \item This led to the development of the Theory of Relativity and Quantum Theory

  \item Some theories of classical physics which helped develop modern physics, include:

    \begin{itemize}

      \item Newton's law of mechanics, which describes interactions among microscopic particles

      \item Maxwell's equations, which unify electricity and magnetism

      \item The laws of thermodynamics

    \end{itemize}

  \item In the early \nth{20} century, two theories emerged:

    \begin{itemize}

      \item Special Theory of Relativity (1905) — Einstein

      \item Quantum Theory (1900) — Planck

    \end{itemize}

  \item Classical Relativity

    \begin{itemize}

      \item A theory of relativity provides a mathematical basis for expressing physical laws in different frames of reference

      \item The mathematical basis is called a transformation

      \item Ex. Two observers, $O$, who is still, and $O'$, who is moving, are at rest in their own frames of reference (FOR). Relative velocity is defined as $\overline{u}$. For this course, an inertial FOR will be used, meaning Newton's law holds, where $v=0$, or constant, unless $\overline{F}\neq0$. $O$ and $O'$ observe the same event.

        \begin{itemize}

          \item Four quantities describe this event for $O$: $x,y,z,t$

          \item For $O'$, these quantities are: $x',y',z',t'$

          \item Assuming postulate: $t=t'$

            \begin{itemize}

              \item Also, at $t=0$, the two origins coincide

            \end{itemize}

          \item To find $x'$ from $x$, this would become $x'= x - ut$

          \item $y'$ and $z'$ remain equal to $y$ and $z$, respectively

          \item This is defined as a Galilean Transformation

          \item As velocity is the first derivative, this yields $\left\{\begin{array}{c} v_x = \dfrac{dx}{dt}\\v_y = \dfrac{dy}{dt}\\ v_z=\dfrac{dz}{dt}\end{array}$ and $\left\{\begin{array}{c} v_{x'} = v_x-u\\v_{y'} = v_y\\ v_{z'}=v_z\end{array}$ for $O$ and $O'$, respectively

                \vspace{10pt}

          \item This means the acceleration components are all equal

        \end{itemize}

    \end{itemize}

  \item Consequences of classical relativity

    \begin{itemize}

      \item From Maxwell's equations, it is concluded that light is an electromagnetic wave

        \begin{itemize}

          \item Light travels in some medium, at speed $\boxed{c=\dfrac{1}{\sqrt{\mu_0\epsilon_0}}}\approx3\times10^8\left[\frac{\si{\meter}}{\si{\second}}\right]$

          \item A postulate from Maxwell is that there is a preferred frame of reference with ``ether'' at rest, in which the speed of light is precisely $c$

          \item Ether — An invisible, massless medium

        \end{itemize}

    \end{itemize}

  \item Michelson-Morley Experiment (1887)

    \begin{figure}[h!]
      \centering
      \tikzset{every picture/.style={line width=0.75pt}} %set default line width to 0.75pt        

\begin{tikzpicture}[x=0.75pt,y=0.75pt,yscale=-1,xscale=1]
%uncomment if require: \path (0,359); %set diagram left start at 0, and has height of 359

%Shape: Rectangle [id:dp0422405126406995] 
\draw   (279.73,92.14) -- (401.36,213.77) -- (382.27,232.86) -- (260.64,111.23) -- cycle ;
%Straight Lines [id:da7482617222343462] 
\draw    (319.71,170) -- (178.29,170) ;
%Shape: Boxed Line [id:dp1498000723380688] 
\draw    (319.71,170) -- (319.71,311.42) ;
%Straight Lines [id:da46701816952051156] 
\draw    (279.86,311.21) -- (359.57,311.63) ;
%Straight Lines [id:da5738747068731038] 
\draw    (327.71,178) -- (327.71,311.42) ;
%Shape: Boxed Line [id:dp7543276922096258] 
\draw    (345.5,158.21) -- (487.92,158.21) ;
%Straight Lines [id:da012479980212346975] 
\draw    (353.5,166.21) -- (486.92,166.21) ;
%Straight Lines [id:da9333868145428461] 
\draw    (487.01,126.36) -- (486.83,206.07) ;
%Shape: Boxed Line [id:dp5760950591204244] 
\draw    (328.21,139.92) -- (327.21,-1.5) ;
%Straight Lines [id:da19075996417646723] 
\draw    (320.21,131.92) -- (320.21,-1.5) ;

% Text Node
\draw (176.29,170) node [anchor=east] [inner sep=0.75pt]  [font=\Large] [align=left] {S};
% Text Node
\draw (329.21,1.5) node [anchor=north west][inner sep=0.75pt]  [font=\Large] [align=left] {O};
% Text Node
\draw (317.71,173) node [anchor=north east] [inner sep=0.75pt]  [font=\Large] [align=left] {A};
% Text Node
\draw (317.71,308.42) node [anchor=south east] [inner sep=0.75pt]  [font=\Large] [align=left] {B};
% Text Node
\draw (484.92,169.21) node [anchor=north east] [inner sep=0.75pt]  [font=\Large] [align=left] {C};


\end{tikzpicture}

      \caption{The Michelson-Morley Setup}
      \label{fig:1}
    \end{figure}

    \begin{itemize}

      \item S is the source, O is an observer, and A, B, and C, are points along the path of light

      \item Generated a ``fringe'' pattern using light and mirrors

      \item Interference or ``fringe'' appears due to phase difference of light

        \begin{itemize}

          \item Path difference: $2|AB-AC|$

          \item Light travels faster through a cross-stream pattern

        \end{itemize}

      \item With the same setup shown, they then rotated the device $90^{\circ}$

        \begin{itemize}

          \item \nth{2} contribution then changes sign

          \item Thus, phase difference changes

          \item Number of fringes was measured

          \item The result: There was no observable change of fringe pattern — the movement of ether was mapped out to be a speed of $u < 5\left[ \frac{\si{\kilo\meter}}{\si{\second}} \right]$

          \item This experiment was redone over the course of many years, most recently Herman at al. (2009), with $u < 10^{-8}\left[ \frac{\si{\centi\meter}}{\si{\second}} \right]$

        \end{itemize}

      \item This indicates that $c$ is a constant, in any inertial reference frame

    \end{itemize}

  \item Einstein's postulates for inertial relativity

    \begin{enumerate}

      \item The principle of relativity — The physical laws are the same in all inertial reference frames

      \item The principle of the constancy of the speed of light — The speed of light in free space has the same value $c$ in all inertial reference frames

    \end{enumerate}

    \begin{itemize}

      \item The second postulate requires observers in all inertial reference frames to measure the same speed of $c$ for the light beam

      \item This explains the failure of Michelson \& Morley

      \item Now we can ``dispose'' of the ether hypothesis

    \end{itemize}

    \begin{enumerate}

      \item \nth{1} postulate doesn't allow a preferred frame of reference where ether stays at rest

      \item \nth{2} postulate doesn't allow only a single frame of reference with light moving at speed $c$

    \end{enumerate}

\end{itemize}

\end{document}

