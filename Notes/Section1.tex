%%%%%%%%%%%%%%%%%%%%%%%%%%%%%%%%%%%%%%%%%%%%%%%%%%%%%%%%%%%%%%%%%%%%%%%%%%%%%%%%%%%%%%%%%%%%%%%%%%%%%%%%%%%%%%%%%%%%%%%%%%%%%%%%%%%%%%%%%%%%%%%%%%%%%%%%%%%%%%%%%%%
% Written By Michael Brodskiy
% Class: Modern Physics
% Professor: Q. Yan
%%%%%%%%%%%%%%%%%%%%%%%%%%%%%%%%%%%%%%%%%%%%%%%%%%%%%%%%%%%%%%%%%%%%%%%%%%%%%%%%%%%%%%%%%%%%%%%%%%%%%%%%%%%%%%%%%%%%%%%%%%%%%%%%%%%%%%%%%%%%%%%%%%%%%%%%%%%%%%%%%%%

\include{Includes.tex}

\title{Introduction to Modern Physics}
\date{\today}
\author{Michael Brodskiy\\ \small Professor: Q. Yan}

\begin{document}

\maketitle

\begin{itemize}

  \item Modern physics is a set of developments that emerged around 1900

  \item This led to the development of the Theory of Relativity and Quantum Theory

  \item Some theories of classical physics which helped develop modern physics, include:

    \begin{itemize}

      \item Newton's law of mechanics, which describes interactions among microscopic particles

      \item Maxwell's equations, which unify electricity and magnetism

      \item The laws of thermodynamics

    \end{itemize}

  \item In the early \nth{20} century, two theories emerged:

    \begin{itemize}

      \item Special Theory of Relativity (1905) — Einstein

      \item Quantum Theory (1900) — Planck

    \end{itemize}

  \item Classical Relativity

    \begin{itemize}

      \item A theory of relativity provides a mathematical basis for expressing physical laws in different frames of reference

      \item The mathematical basis is called a transformation

      \item Ex. Two observers, $O$, who is still, and $O'$, who is moving, are at rest in their own frames of reference (FOR). Relative velocity is defined as $\overline{u}$. For this course, an inertial FOR will be used, meaning Newton's law holds, where $v=0$, or constant, unless $\overline{F}\neq0$. $O$ and $O'$ observe the same event.

        \begin{itemize}

          \item Four quantities describe this event for $O$: $x,y,z,t$

          \item For $O'$, these quantities are: $x',y',z',t'$

          \item Assuming postulate: $t=t'$

            \begin{itemize}

              \item Also, at $t=0$, the two origins coincide

            \end{itemize}

          \item To find $x'$ from $x$, this would become $x'= x - ut$

          \item $y'$ and $z'$ remain equal to $y$ and $z$, respectively

          \item This is defined as a Galilean Transformation

          \item As velocity is the first derivative, this yields $\left\{\begin{array}{c} v_x = \dfrac{dx}{dt}\\v_y = \dfrac{dy}{dt}\\ v_z=\dfrac{dz}{dt}\end{array}$ and $\left\{\begin{array}{c} v_{x'} = v_x-u\\v_{y'} = v_y\\ v_{z'}=v_z\end{array}$ for $O$ and $O'$, respectively

                \vspace{10pt}

          \item This means the acceleration components are all equal

        \end{itemize}

    \end{itemize}

\end{itemize}

\end{document}

