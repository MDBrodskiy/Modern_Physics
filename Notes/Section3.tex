%%%%%%%%%%%%%%%%%%%%%%%%%%%%%%%%%%%%%%%%%%%%%%%%%%%%%%%%%%%%%%%%%%%%%%%%%%%%%%%%%%%%%%%%%%%%%%%%%%%%%%%%%%%%%%%%%%%%%%%%%%%%%%%%%%%%%%%%%%%%%%%%%%%%%%%%%%%%%%%%%%%
% Written By Michael Brodskiy
% Class: Modern Physics
% Professor: Q. Yan
%%%%%%%%%%%%%%%%%%%%%%%%%%%%%%%%%%%%%%%%%%%%%%%%%%%%%%%%%%%%%%%%%%%%%%%%%%%%%%%%%%%%%%%%%%%%%%%%%%%%%%%%%%%%%%%%%%%%%%%%%%%%%%%%%%%%%%%%%%%%%%%%%%%%%%%%%%%%%%%%%%%

\include{Includes.tex}

\title{The Wave-Like Properties of Particles}
\date{\today}
\author{Michael Brodskiy\\ \small Professor: Q. Yan}

\begin{document}

\maketitle

\newpage

\tableofcontents

\newpage

\begin{itemize}

    \section{De Broglie's Hypothesis}

  \item After Einstein's theory, it was determined that light has dual particle-wave nature

  \item In 1924, Louis de Broglie proposes a hypothesis:

    \begin{itemize}

      \item Any object moving with a momentum $p$ is associated with a wave of wavelength $\lambda$, where:

        $$\boxed{\lambda=\frac{h}{p}}$$

      \item $\lambda$ refers to the ``De Broglie'' wavelength, $h$ is the Planck constant, and $p$ is the momentum

      \item For experimental measurement of the wave--like behavior of particles, the double and single-slit experiments were performed

    \end{itemize}

    \section{Experimental Evidence for De Broglie Waves}

  \item Particle Diffraction Experiment

    \begin{itemize}

      \item For light of wavelength $\lambda$ incident on a slit of width $a$, the diffraction pattern has a minimum at angles:

        $$\boxed{a\sin(\theta)=n\lambda,\,\,\,\,\,\,\,\,\, n=1,2,3,\cdots}$$

      \item Each of the atoms acts as a scatter

      \item The scattered electron waves interfere

      \item The crystal serves as a diffraction grating

      \item The maxima occurs at angle:

        $$\boxed{d\sin(\phi)=n\lambda}$$

      \item Where $\lambda$ is the de Broglie wavelength

    \end{itemize}

\end{itemize}

\end{document}

