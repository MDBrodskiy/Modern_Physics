%%%%%%%%%%%%%%%%%%%%%%%%%%%%%%%%%%%%%%%%%%%%%%%%%%%%%%%%%%%%%%%%%%%%%%%%%%%%%%%%%%%%%%%%%%%%%%%%%%%%%%%%%%%%%%%%%%%%%%%%%%%%%%%%%%%%%%%%%%%%%%%%%%%%%%%%%%%%%%%%%%%
% Written By Michael Brodskiy
% Class: Modern Physics
% Professor: Q. Yan
%%%%%%%%%%%%%%%%%%%%%%%%%%%%%%%%%%%%%%%%%%%%%%%%%%%%%%%%%%%%%%%%%%%%%%%%%%%%%%%%%%%%%%%%%%%%%%%%%%%%%%%%%%%%%%%%%%%%%%%%%%%%%%%%%%%%%%%%%%%%%%%%%%%%%%%%%%%%%%%%%%%

\documentclass[12pt]{article} 
\usepackage{alphalph}
\usepackage[utf8]{inputenc}
\usepackage[russian,english]{babel}
\usepackage{titling}
\usepackage{amsmath}
\usepackage{graphicx}
\usepackage{enumitem}
\usepackage{amssymb}
\usepackage[super]{nth}
\usepackage{everysel}
\usepackage{ragged2e}
\usepackage{geometry}
\usepackage{multicol}
\usepackage{fancyhdr}
\usepackage{cancel}
\usepackage{siunitx}
\usepackage{physics}
\usepackage{tikz}
\usepackage{mathdots}
\usepackage{yhmath}
\usepackage{cancel}
\usepackage{color}
\usepackage{array}
\usepackage{multirow}
\usepackage{gensymb}
\usepackage{tabularx}
\usepackage{extarrows}
\usepackage{booktabs}
\usepackage{expl3}
\usepackage[version=4]{mhchem}
\usepackage{hpstatement}
\usetikzlibrary{fadings}
\usetikzlibrary{patterns}
\usetikzlibrary{shadows.blur}
\usetikzlibrary{shapes}

\geometry{top=1.0in,bottom=1.0in,left=1.0in,right=1.0in}
\newcommand{\subtitle}[1]{%
  \posttitle{%
    \par\end{center}
    \begin{center}\large#1\end{center}
    \vskip0.5em}%

}
\usepackage{hyperref}
\hypersetup{
colorlinks=true,
linkcolor=blue,
filecolor=magenta,      
urlcolor=blue,
citecolor=blue,
}


\title{The Wave-Like Properties of Particles}
\date{\today}
\author{Michael Brodskiy\\ \small Professor: Q. Yan}

\begin{document}

\maketitle

\newpage

\tableofcontents

\newpage

\begin{itemize}

    \section{De Broglie's Hypothesis}

  \item After Einstein's theory, it was determined that light has dual particle-wave nature

  \item In 1924, Louis de Broglie proposes a hypothesis:

    \begin{itemize}

      \item Any object moving with a momentum $p$ is associated with a wave of wavelength $\lambda$, where:

        $$\boxed{\lambda=\frac{h}{p}}$$

      \item $\lambda$ refers to the ``De Broglie'' wavelength, $h$ is the Planck constant, and $p$ is the momentum

      \item For experimental measurement of the wave--like behavior of particles, the double and single-slit experiments were performed

    \end{itemize}

    \section{Experimental Evidence for De Broglie Waves}

  \item Particle Diffraction Experiment

    \begin{itemize}

      \item For light of wavelength $\lambda$ incident on a slit of width $a$, the diffraction pattern has a minimum at angles:

        $$\boxed{a\sin(\theta)=n\lambda,\,\,\,\,\,\,\,\,\, n=1,2,3,\cdots}$$

      \item Each of the atoms acts as a scatter

      \item The scattered electron waves interfere

      \item The crystal serves as a diffraction grating

      \item The maxima occurs at angle:

        $$\boxed{d\sin(\phi)=n\lambda}$$

      \item Where $\lambda$ is the de Broglie wavelength

    \end{itemize}

    \subsection{Double-Slit Experiment}

  \item Question: Through which slit does the particle pass?

  \item Result: No diffraction pattern on the screen

  \item if we check which slit the particle passes through:

    \begin{itemize}

      \item Particle behavior is measured

      \item We can not observe its wave nature simultaneously! (Principle of complementarity)

    \end{itemize}

  \item Conclusion: 

    \begin{itemize}

      \item The electron will behave as a wave or a particle

    \end{itemize}

    \section{Heisenberg Uncertainty Relationships}

  \item Applying the uncertainty relationship to de Broglie waves:

    $$p=\frac{h}{\lambda}\Rightarrow dp=-\frac{h}{\lambda^2}\,d\lambda\Rightarrow \Delta p = \frac{h}{\lambda^2}\Delta\lambda$$

  \item Finally, this yields:

    $$\Delta x\Delta p\approx \varepsilon h$$

  \item From quantum mechanics:

    $$\Delta x\Delta p\geq\frac{h}{4\pi}$$

    $$\varepsilon=\frac{1}{4\pi}$$

    $$\Delta x \Delta p \geq \frac{1}{2}\hbar$$

  \item Where $\hbar=\dfrac{h}{2\pi}$

  \item When a coin is flipped, or a dice is rolled:

    \begin{itemize}

      \item No way to predict a single flip/roll

      \item But, we can predict the distribution of the results from a large \# of flips or rolls

      \item Quantum Theory allows for the same behavior

    \end{itemize}
    
  \item Wave Function

    \begin{itemize}

      \item What is the amplitude of the de Broglie wave?

      \item Checking classical waves:

        \begin{itemize}

          \item Waves in the ocean: Height of water level

          \item Sound wave: Volume density of molecules

          \item Light waves: $\overrightarrow{E},\overrightarrow{B}$ field

          \item de Broglie waves: The probability of finding a particle at a given $(x,t)$

            \begin{itemize}

              \item This is known as $\psi$, the wave function

              \item In $n$-dimensional space, it becomes $\psi(x_1,x_2,\cdots,x_n,t)$

              \item In classical physics, the intensity ($I$) of any wave is proportional to $|A|$

              \item For quantum mechanics, we have the probability of final particle $P\propto |\psi|^2$

            \end{itemize}

          \item The requirement for wave function $\psi$ is that $|\psi|^2\geq0$

          \item Any physical measurement is related to $P\propto |\psi|^2$

          \item $\psi$ are generally complex \#'s

          \item Properties of Complex Numbers

            \begin{itemize}

              \item $\psi = Re(\psi) + iIm(\psi)$

              \item The complex conjugate is: $\psi^* = Re(\psi) - iIm(\psi)$

            \end{itemize}

        \end{itemize}

    \end{itemize}

  \item In the complex plane:

    \begin{itemize}

      \item The phase factor is $\boxed{z=|z|e^{i\theta}}$

      \item The wave function of a free particle is $Ae^{i(kx-\omega t)}$ or $A[\cos(kx-\omega t)+i]sin(kx-\omega t)]$
        
    \end{itemize}

  \item Behavior of a wave function:

    \begin{itemize}

      \item Reflection and transmission at a boundary

      \item Penetration of the reflected wave

      \item Continuity at the boundary

    \end{itemize}

  \item The mathematical solution for wave functions

    \begin{itemize}

      \item The wave function itself must be continuous

      \item The slope of the wave function must be continuous (``boundary condition'')

      \item By confing a particle by 2 boundaries, we've learned:

        \begin{itemize}

          \item It may be anywhere in space

          \item A definite/continuous valued $\lambda,p,E$

        \end{itemize}

    \end{itemize}

  \item In the left/right region:

    $$U=qV=(-e)(-V_o)=eV_o$$

  \item A wave confined in a well with infinite-height barriers is known as a standing wave

  \item The two nodes ($A=0$) are at two boundaries

  \item With this standing wave:

    $$\lambda_n=\frac{2L}{n-1},\,\,\,\,\,\,\,\,\,\,n=1,2,3\cdots$$

  \item Where $n$ is the number of nodes, and $L$ is the width of the well

  \item From de Broglie theory, the energy becomes:

    $$E_n=n^2\left( \frac{h^2}{8mL^2} \right)$$

  \item Where the expression in the parenthesis is $E_o$

  \item The $n$ term makes it so that energy is quantized

  \item This is the general nature of quantum particles

  \item A particle confined in space $\to$ energy is quantized

  \item Schr\"odinger's Equation:

    \begin{itemize}

      \item The behavior of the wave function is controlled by a differential eqution called ``Schr\"odinger's equation''

      \item The role is similar to Newton's \nth{2} law

      \item It is a second order differential equation

      \item Using a free particle, with $A$ as the amplitude, to derive the equation (at a given time $t=t_o$), $\psi(x)$ is:

        $$\boxed{\psi(x)=A\sin(kx),\,\,\,\,\,\,\,\,\,\,k=\frac{2\pi}{\lambda}}$$

        $$\boxed{\frac{d\psi(x)}{dx}=Ak\cos(kx)}$$

        $$\boxed{\frac{d^2\psi(x)}{dx}=-Ak^2\sin(kx)=-k^2\psi(x)}$$

      \item Using the kinetic energy, $K=\dfrac{p^2}{2m}=\left( \dfrac{\hbar}{\lambda} \right)^2\dfrac{1}{2m}$

        $$K=\frac{\hbar^2k^2}{2m}$$

      \item Thus, we obtain:

        $$\boxed{-\frac{\hbar^2}{2m}\frac{d^2\psi(x)}{dx^2}=K\psi(x)}$$

      \item Using total energy:

        $$\boxed{-\frac{\hbar^2}{2m}\frac{d^2\psi(x)}{dx^2}+U(x)\psi(x)=E\psi(x)}$$

      \item This is the unidimensional, static\footnote{time independent} Schr\"odinger equation

    \end{itemize}

  \item Probability density and normalization

    \begin{itemize}

      \item We defined $P(x)\,dx=|\psi(x)|^2\,dx$

      \item $P(x)\,dx$ represents the probability density


        $$\int_{x_1}^{x_2} P(x)\,dx\Rightarrow P(x_1:x_2)$$

      \item Is the probability of finding the particle in range $x_1$ to $x_2$

        $$\int_{-\infty}^{\infty}P(x)\,dx=1\,\,\forall\text{ particles}$$

      \item This means:

        $$\boxed{\int_{-\infty}^{\infty} |\psi(x)|^2\,dx = 1}$$

      \item The average location of the particle is given by:

        $$\boxed{\dfrac{\sum n_1x_1+n_2x_2+\cdots+n_ix_i}{\sum n_1+n_2+\cdots+n_i}}$$

      \item On a much smaller interval, we can use:

        $$x_{avg}=\dfrac{\displaystyle \int_{-\infty}^{\infty} xP(x)\,dx}{\displaystyle \int_{-\infty}^{\infty} P(x)\,dx}=\int_{-\infty}^{\infty} x|\psi(x)|^2\,dx$$

      \item For any function of $x$, the average is:

        $$[f(x)]_{avg}=\int_{-\infty}^{\infty}f(x)|\psi(x)|^2\,dx$$

    \end{itemize}

  \item Given a case of constant potential energy:

    $$\psi(x)=A\sin(kx)+B\cos(kx)$$

  \item Given a case where $E < U_o$\footnote{$e^{kx}$ has to be removed because it diverges}

    $$\psi(x)=Ae^{-kx}$$

\end{itemize}

\end{document}

