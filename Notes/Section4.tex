%%%%%%%%%%%%%%%%%%%%%%%%%%%%%%%%%%%%%%%%%%%%%%%%%%%%%%%%%%%%%%%%%%%%%%%%%%%%%%%%%%%%%%%%%%%%%%%%%%%%%%%%%%%%%%%%%%%%%%%%%%%%%%%%%%%%%%%%%%%%%%%%%%%%%%%%%%%%%%%%%%%
% Written By Michael Brodskiy
% Class: Modern Physics
% Professor: Q. Yan
%%%%%%%%%%%%%%%%%%%%%%%%%%%%%%%%%%%%%%%%%%%%%%%%%%%%%%%%%%%%%%%%%%%%%%%%%%%%%%%%%%%%%%%%%%%%%%%%%%%%%%%%%%%%%%%%%%%%%%%%%%%%%%%%%%%%%%%%%%%%%%%%%%%%%%%%%%%%%%%%%%%

\documentclass[12pt]{article} 
\usepackage{alphalph}
\usepackage[utf8]{inputenc}
\usepackage[russian,english]{babel}
\usepackage{titling}
\usepackage{amsmath}
\usepackage{graphicx}
\usepackage{enumitem}
\usepackage{amssymb}
\usepackage[super]{nth}
\usepackage{everysel}
\usepackage{ragged2e}
\usepackage{geometry}
\usepackage{multicol}
\usepackage{fancyhdr}
\usepackage{cancel}
\usepackage{siunitx}
\usepackage{physics}
\usepackage{tikz}
\usepackage{mathdots}
\usepackage{yhmath}
\usepackage{cancel}
\usepackage{color}
\usepackage{array}
\usepackage{multirow}
\usepackage{gensymb}
\usepackage{tabularx}
\usepackage{extarrows}
\usepackage{booktabs}
\usepackage{expl3}
\usepackage[version=4]{mhchem}
\usepackage{hpstatement}
\usetikzlibrary{fadings}
\usetikzlibrary{patterns}
\usetikzlibrary{shadows.blur}
\usetikzlibrary{shapes}

\geometry{top=1.0in,bottom=1.0in,left=1.0in,right=1.0in}
\newcommand{\subtitle}[1]{%
  \posttitle{%
    \par\end{center}
    \begin{center}\large#1\end{center}
    \vskip0.5em}%

}
\usepackage{hyperref}
\hypersetup{
colorlinks=true,
linkcolor=blue,
filecolor=magenta,      
urlcolor=blue,
citecolor=blue,
}


\title{Schr\"odinger's Equation}
\date{\today}
\author{Michael Brodskiy\\ \small Professor: Q. Yan}

\begin{document}

\maketitle

\newpage

\tableofcontents

\newpage

\begin{itemize}

    \section{The Schr\"odinger Equation}

  \item Schr\"odinger's Equation:

    \begin{itemize}

      \item The behavior of the wave function is controlled by a differential eqution called ``Schr\"odinger's equation''

      \item The role is similar to Newton's \nth{2} law

      \item It is a second order differential equation

      \item Using a free particle, with $A$ as the amplitude, to derive the equation (at a given time $t=t_o$), $\psi(x)$ is:

        $$\boxed{\psi(x)=A\sin(kx),\,\,\,\,\,\,\,\,\,\,k=\frac{2\pi}{\lambda}}$$

        $$\boxed{\frac{d\psi(x)}{dx}=Ak\cos(kx)}$$

        $$\boxed{\frac{d^2\psi(x)}{dx}=-Ak^2\sin(kx)=-k^2\psi(x)}$$

      \item Using the kinetic energy, $K=\dfrac{p^2}{2m}=\left( \dfrac{\hbar}{\lambda} \right)^2\dfrac{1}{2m}$

        $$K=\frac{\hbar^2k^2}{2m}$$

      \item Thus, we obtain:

        $$\boxed{-\frac{\hbar^2}{2m}\frac{d^2\psi(x)}{dx^2}=K\psi(x)}$$

      \item Using total energy:

        $$\boxed{-\frac{\hbar^2}{2m}\frac{d^2\psi(x)}{dx^2}+U(x)\psi(x)=E\psi(x)}$$

      \item This is the unidimensional, static\footnote{time independent} Schr\"odinger equation

    \end{itemize}

  \item Probability density and normalization

    \begin{itemize}

      \item We defined $P(x)\,dx=|\psi(x)|^2\,dx$

      \item $P(x)\,dx$ represents the probability density


        $$\int_{x_1}^{x_2} P(x)\,dx\Rightarrow P(x_1:x_2)$$

      \item Is the probability of finding the particle in range $x_1$ to $x_2$

        $$\int_{-\infty}^{\infty}P(x)\,dx=1\,\,\forall\text{ particles}$$

      \item This means:

        $$\boxed{\int_{-\infty}^{\infty} |\psi(x)|^2\,dx = 1}$$

      \item The average location of the particle is given by:

        $$\boxed{\dfrac{\sum n_1x_1+n_2x_2+\cdots+n_ix_i}{\sum n_1+n_2+\cdots+n_i}}$$

      \item On a much smaller interval, we can use:

        $$x_{avg}=\dfrac{\displaystyle \int_{-\infty}^{\infty} xP(x)\,dx}{\displaystyle \int_{-\infty}^{\infty} P(x)\,dx}=\int_{-\infty}^{\infty} x|\psi(x)|^2\,dx$$

      \item For any function of $x$, the average is:

        $$[f(x)]_{avg}=\int_{-\infty}^{\infty}f(x)|\psi(x)|^2\,dx$$

    \end{itemize}

  \item Given a case of constant potential energy:

    $$\psi(x)=A\sin(kx)+B\cos(kx)$$

  \item Given a case where $E < U_o$\footnote{$e^{kx}$ has to be removed because it diverges}

    $$\psi(x)=Ae^{-kx}$$

  \item Light emission or absorption with quantum wave systems

    \begin{itemize}

      \item By emitting energy in photons, the particle can move from a higher state to a lower state

      \item By absorbing energy from photons, the particle can move from a lower to a higher state

    \end{itemize}

  \item Infinite vs. Finite Wells

    \begin{itemize}

      \item Unlike the infinite well, the particle can penetrate the forbidden region

      \item Excited states are more likely to penetrate deeper into the forbidden region

    \end{itemize}

  \item Two-Dimensional Quantum Wells

    \begin{itemize}

      \item The potential energy in an infinite potential energy well:

        $$U(x,y)=\left\{\begin{array}{l r} 0, & 0\leq (x,y) \leq L\\ \infty, & \text{otherwise} \end{array}$$

        \item Inside the quantum well:

          $$\psi(x,y)=f(x)g(y)$$

        \item The boundary conditions are:

          $$\begin{array}{l r} \psi(0,y)=0 & \psi(L,y)=0\\ \psi(x,0)=0 & \psi(x,L)=0\end{array}$$

        \item This results in the function:

          $$\psi(x,y)=A\sin\left( \dfrac{n\pi x}{L} \right)\sin\left( \dfrac{n\pi y}{L} \right)$$

        \item Using the property of the wave function, we obtain:

          $$\int_0^L\int_0^L |\psi(x)|^2\,dx\,dy=1$$
          $$A=\frac{2}{L}$$

        \item This results in:

          $$E_n=\frac{\hbar^2\pi^2}{2mL^2}(n_x^2+n_y^2)$$

    \end{itemize}

\end{itemize}

\end{document}

