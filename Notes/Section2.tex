%%%%%%%%%%%%%%%%%%%%%%%%%%%%%%%%%%%%%%%%%%%%%%%%%%%%%%%%%%%%%%%%%%%%%%%%%%%%%%%%%%%%%%%%%%%%%%%%%%%%%%%%%%%%%%%%%%%%%%%%%%%%%%%%%%%%%%%%%%%%%%%%%%%%%%%%%%%%%%%%%%%
% Written By Michael Brodskiy
% Class: Modern Physics
% Professor: Q. Yan
%%%%%%%%%%%%%%%%%%%%%%%%%%%%%%%%%%%%%%%%%%%%%%%%%%%%%%%%%%%%%%%%%%%%%%%%%%%%%%%%%%%%%%%%%%%%%%%%%%%%%%%%%%%%%%%%%%%%%%%%%%%%%%%%%%%%%%%%%%%%%%%%%%%%%%%%%%%%%%%%%%%

\include{Includes.tex}

\title{The Particle-Like Properties of Electromagnetic Waves}
\date{\today}
\author{Michael Brodskiy\\ \small Professor: Q. Yan}

\begin{document}

\maketitle

\newpage

\tableofcontents

\newpage

\begin{itemize}

    \section{Review of Electromagnetic Waves}

  \item Reviewing the nature of light (or electromagnetic waves)

    \begin{itemize}

      \item A plane wave

        \begin{itemize}

          \item A plane wave traveling in the positive x direction:

            $$\left\{\begin{array}{c} \vec{E}=\vec{E}_o\sin(kx-\omega t)\\\vec{B}=\vec{B}_o\sin(kx-\omega t) \end{array}$$

            \item The direction of energy transport would be $\vec{E}\times\vec{B}$

            \item $|\vec{E}|$ and $|\vec{B}|$ are constant at a given $t$

            \item The power, $P$, of the wave:

              $$P=\frac{E}{\Delta t}=\frac{1}{\mu_oc}E_o^2A\sin^2(kx-\omega t)$$

            \item Two important features:

              \begin{enumerate}

                \item Intensity (average power per unit area) is proportional to $E_o^2$

                  $$P_{avg}=\int_0^T P(t)\,dt$$

                \item  The intensity of the system fluctuates with time

                  $$\frac{P_{avg}}{A}$$

              \end{enumerate}

        \end{itemize}

      \item A spherical wave

        \begin{itemize}

          \item Spreads out uniformly along the three axis

        \end{itemize}

    \end{itemize}

    \section{The Photoelectric Effect}

  \item Experiment performed by Heinrich Hertz (1887)

  \item When a metal surface is illuminated, light electrons can be emitted from the surface

  \item The Experiment:

    \begin{itemize}

      \item Connect emitter and collector to an external circuit

      \item Apply a negative potential to the circuit collector

      \item Increase the potential difference ($(-V)-(+V)$) to be more negative

      \item At some point, even the most energetic electrons do not have enough kinetic energy to reach the collector

      \item The maximum kinetic energy to reach the collector with the stopping voltage, $V_s$, is:

        $$K_{max}=eV_s$$

    \end{itemize}

    \begin{figure}[h!]
      \centering
      \tikzset{every picture/.style={line width=0.75pt}} %set default line width to 0.75pt        

\begin{tikzpicture}[x=0.75pt,y=0.75pt,yscale=-1,xscale=1]
%uncomment if require: \path (0,642); %set diagram left start at 0, and has height of 642

%Shape: Boxed Line [id:dp5098624474614848] 
\draw    (150,108.29) -- (150,249.71) ;
%Shape: Boxed Line [id:dp4883126832478788] 
\draw    (452,108.29) -- (452,249.71) ;
%Shape: Battery [id:dp21984735994344673] 
\draw   (453,249.71) -- (316.65,249.71) (286.35,279.71) -- (286.35,219.71) (286.35,249.71) -- (150,249.71) (328.77,264.71) -- (316.65,264.71) -- (316.65,234.71) -- (328.77,234.71) -- (328.77,264.71) -- cycle ;
%Straight Lines [id:da07825154591470551] 
\draw    (240.42,108.29) -- (150,108.29) ;
%Straight Lines [id:da10296831027951714] 
\draw    (452,108.29) -- (361.58,108.29) ;
%Shape: Rectangle [id:dp45994932364238905] 
\draw   (240,78) -- (262,78) -- (262,139) -- (240,139) -- cycle ;
%Straight Lines [id:da8703604595072303] 
\draw    (361.58,37.58) -- (361.58,179) ;
%Straight Lines [id:da37814010167699585] 
\draw  [dash pattern={on 0.84pt off 2.51pt}]  (262.42,101.29) -- (362.42,1.29) ;
%Straight Lines [id:da7637770505032595] 
\draw  [dash pattern={on 0.84pt off 2.51pt}]  (262.42,108.29) -- (362.42,8.29) ;
%Straight Lines [id:da7153723628671289] 
\draw  [dash pattern={on 0.84pt off 2.51pt}]  (261.42,95.29) -- (361.42,-4.71) ;
%Straight Lines [id:da04025150844205161] 
\draw    (262.42,108.29) -- (298.58,108.29) ;
\draw [shift={(300.58,108.29)}, rotate = 180] [color={rgb, 255:red, 0; green, 0; blue, 0 }  ][line width=0.75]    (10.93,-3.29) .. controls (6.95,-1.4) and (3.31,-0.3) .. (0,0) .. controls (3.31,0.3) and (6.95,1.4) .. (10.93,3.29)   ;

% Text Node
\draw (284.35,282.71) node [anchor=north east] [inner sep=0.75pt]   [align=left] {$\displaystyle +V$};
% Text Node
\draw (330.77,267.71) node [anchor=north west][inner sep=0.75pt]   [align=left] {$\displaystyle -V$};
% Text Node
\draw (238,75) node [anchor=south east] [inner sep=0.75pt]   [align=left] {Emitter};
% Text Node
\draw (300.58,111.29) node [anchor=north] [inner sep=0.75pt]   [align=left] {$\displaystyle e^{-}$};


\end{tikzpicture}

      \caption{Set up of the Photoelectric Effect Experiment}
      \label{fig:1}
    \end{figure}

    \begin{itemize}

      \item The classical picture: The energy of light with intensity $I$ is absorbed by electrons, $E_{light} > E_{binding}$, $e$ is released


      \item What does the classical wave theory predict?

        \begin{enumerate}

          \item The maximum kinetic energy of the electrons, $K_{max}$, is proportional to the intensity of light

          \item The effect occurs for light with any frequency or wavelength

          \item $e^-$ are released after a finite $\Delta t$

        \end{enumerate}

    \end{itemize}

  \item Experimental Results

    \begin{enumerate}

      \item For a fixed $f$ or $\lambda$, $K_{max}$ is independent of the intensity of light

      \item The effect occurs only if $f>f_{\text{cutoff}}$

      \item The first electrons are emitted almost instantaneously ($<10^{-9}[\si{\second}]$)

    \end{enumerate}

    \begin{itemize}

      \item This means everything that classical wave theory predicted was essentially incorrect

    \end{itemize}

  \item The Quantum Theory of the Photoelectric Effect

    \begin{itemize}

      \item Developed by Albert Einstein (in 1905), based on Max Planck's idea explaining thermal radiation

      \item Assumptions:

        \begin{itemize}

          \item The energy of electromagnetic waves is not continuously distributed

          \item The energy is concentrated in localized bands or ``quanta''

          \item This quanta is called ``photon''

        \end{itemize}

      \item The energy of a photon is $\boxed{E=hf}$, where $h$ is Planck's constant, and $f$ is the frequency

        $$\boxed{f=\frac{c}{\lambda}\Rightarrow E=\frac{hc}{\lambda}}$$

      \item Photons travel at speed $c$, and are technically massless, so:

        $$\boxed{p=\frac{E}{c}=\frac{h}{\lambda}}$$

      \item If $E=hf>\phi$, then photoelectrons are released; $E$ is the photon energy, and $\phi$ is the work function

      \item The kinetic energy of the electron is:

        $$\boxed{K_{max}=hf-\phi}$$

      \item Evidently, the intensity is not relevant; a larger intensity would mean more photons in a unit area, which means more electrons released; this means there is more current.

    \end{itemize}

  \item In 1915, Robert Millikan performed an experiment (won 1923 Nobel Prize)

    \begin{itemize}

      \item Determined Planck's constant ($h=6.57\cdot10^{-34}\si{\joule\second}$)

      \item Fairly accurate, modern calculations found $h=6.626\cdot10^{-34} \si{\joule\second}$

    \end{itemize}

\end{itemize}

\end{document}

