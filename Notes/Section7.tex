%%%%%%%%%%%%%%%%%%%%%%%%%%%%%%%%%%%%%%%%%%%%%%%%%%%%%%%%%%%%%%%%%%%%%%%%%%%%%%%%%%%%%%%%%%%%%%%%%%%%%%%%%%%%%%%%%%%%%%%%%%%%%%%%%%%%%%%%%%%%%%%%%%%%%%%%%%%%%%%%%%%
% Written By Michael Brodskiy
% Class: Modern Physics
% Professor: Q. Yan
%%%%%%%%%%%%%%%%%%%%%%%%%%%%%%%%%%%%%%%%%%%%%%%%%%%%%%%%%%%%%%%%%%%%%%%%%%%%%%%%%%%%%%%%%%%%%%%%%%%%%%%%%%%%%%%%%%%%%%%%%%%%%%%%%%%%%%%%%%%%%%%%%%%%%%%%%%%%%%%%%%%

\include{Includes.tex}

\title{Many-Electron Atoms}
\date{\today}
\author{Michael Brodskiy\\ \small Professor: Q. Yan}

\begin{document}

\maketitle

\newpage

\tableofcontents

\newpage

\begin{itemize}

    \section{The Pauli Exclusion Principle}

  \item An important rule proposed by Wolfgang Pauli (1925):

    \begin{itemize}

      \item No two electrons in a single atom can have the same set of quantum numbers $(n,l,m_l,m_s)$

      \item It applies to all ``spin 1/2'' particles (fermions)

    \end{itemize}

  \item Examples:

    \begin{itemize}

      \item Hydrogen: $1e^-$ in ground state: $\left(1,0,0,\pm\frac{1}{2}\right)$

      \item Helium: $2e^-$: $\left(1,0,0,-\frac{1}{2}\right)$ and $\left(1,0,0,\frac{1}{2}\right)$

      \item Lithium: $\left(1,0,0,-\frac{1}{2}\right)$, $\left(1,0,0,\frac{1}{2}\right)$, and $\left( 2,l,m_l,m_s \right)$

        \begin{itemize}

          \item If the electron has spin 1, this may be different:

          \item Lithium: $\left(1,0,0,1\right)$, $\left(1,0,0,\pm1/0\right)$, and $\left( 1,0,0,\pm1/0 \right)$

        \end{itemize}

    \end{itemize}

  \item Electron states in many-electron atoms

    \begin{itemize}

      \item ``Filling rule'': $e^-$'s occupy the lowest levels first

      \item Orbitals with the same $n$ lie at about the same distance from the nucleus $\Longrightarrow r_n=n^2a_o$ (an atomic shell)

        \begin{center}
          \begin{tabular}[h!]{c c c c c c}
            $n$ & 1 & 2 & 3 & 4 & 5 \\
            Shell & $K$ & $L$ & $M$ & $N$ & $O$\\
          \end{tabular}
        \end{center}

      \item According to the Pauli Exclusion Principle, the maximum amount of electrons in each subshell is $2(2l+1)$

      \item Equivalent levels of $d$ are much higher in energy levels because of the ``electron screening effect''

    \end{itemize}

    \section{Outer Electrons: Screening and Optical Transitions}

  \item Screening Effect of Electron Levels

    \begin{itemize}

      \item Lithium ($1s^22s^1$)

        \begin{itemize}

          \item The ionization energy of Li is only $5.39[\si{\eV}]$

          \item For the electron in the $2s$ shell, its ionization energy is $3.4[\si{\eV}]$

          \item This is due to interactions between different shells

        \end{itemize}

      \item To an outer electron, the charge of the nucleus can be screened or shielded by the electrons in the inner shells — this is the screening effect

      \item The less penetrating the wave function, the more accurate the screening model is

      \item The parity of wave functions: 

        $$\begin{array}{l r} \text{Even: }&\psi(x) = \psi(-x)\\ \text{Odd: }&\psi(-x) = -\psi(x) \end{array}$$

    \end{itemize}

  \itemm The Selection Rule

    \begin{itemize}

      \item $\Delta l = l_2 - l_1 = \pm 1$

      \item This means, by optical transition, it is forbidden for an electron to make a transition to a different-numbered, but same-lettered subshell

        \begin{itemize}

          \item For example, $np$ to $ns$ or vice versa is permitted for any values $n$

        \end{itemize}

    \end{itemize}

\end{itemize}

\end{document}

