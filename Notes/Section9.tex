%%%%%%%%%%%%%%%%%%%%%%%%%%%%%%%%%%%%%%%%%%%%%%%%%%%%%%%%%%%%%%%%%%%%%%%%%%%%%%%%%%%%%%%%%%%%%%%%%%%%%%%%%%%%%%%%%%%%%%%%%%%%%%%%%%%%%%%%%%%%%%%%%%%%%%%%%%%%%%%%%%%
% Written By Michael Brodskiy
% Class: Modern Physics
% Professor: Q. Yan
%%%%%%%%%%%%%%%%%%%%%%%%%%%%%%%%%%%%%%%%%%%%%%%%%%%%%%%%%%%%%%%%%%%%%%%%%%%%%%%%%%%%%%%%%%%%%%%%%%%%%%%%%%%%%%%%%%%%%%%%%%%%%%%%%%%%%%%%%%%%%%%%%%%%%%%%%%%%%%%%%%%

\documentclass[12pt]{article} 
\usepackage{alphalph}
\usepackage[utf8]{inputenc}
\usepackage[russian,english]{babel}
\usepackage{titling}
\usepackage{amsmath}
\usepackage{graphicx}
\usepackage{enumitem}
\usepackage{amssymb}
\usepackage[super]{nth}
\usepackage{everysel}
\usepackage{ragged2e}
\usepackage{geometry}
\usepackage{multicol}
\usepackage{fancyhdr}
\usepackage{cancel}
\usepackage{siunitx}
\usepackage{physics}
\usepackage{tikz}
\usepackage{mathdots}
\usepackage{yhmath}
\usepackage{cancel}
\usepackage{color}
\usepackage{array}
\usepackage{multirow}
\usepackage{gensymb}
\usepackage{tabularx}
\usepackage{extarrows}
\usepackage{booktabs}
\usepackage{expl3}
\usepackage[version=4]{mhchem}
\usepackage{hpstatement}
\usetikzlibrary{fadings}
\usetikzlibrary{patterns}
\usetikzlibrary{shadows.blur}
\usetikzlibrary{shapes}

\geometry{top=1.0in,bottom=1.0in,left=1.0in,right=1.0in}
\newcommand{\subtitle}[1]{%
  \posttitle{%
    \par\end{center}
    \begin{center}\large#1\end{center}
    \vskip0.5em}%

}
\usepackage{hyperref}
\hypersetup{
colorlinks=true,
linkcolor=blue,
filecolor=magenta,      
urlcolor=blue,
citecolor=blue,
}

\DeclareSIUnit{\Curie}{Ci}

\title{Nuclear Physics}
\date{\today}
\author{Michael Brodskiy\\ \small Professor: Q. Yan}

\begin{document}

\maketitle

\newpage

\tableofcontents

\newpage

\begin{itemize}

    \section{Nuclear Structure}

  \item The size of an atom is $1[\si{\angstrom}]$

  \item The size of the nucleus is $.001[\si{\angstrom}]$, or $1[\si{\femto\meter}]$

  \item The repulsive positive charges in the nucleus are held together by the strong nuclear force

    \begin{itemize}

      \item Has a very short range

    \end{itemize}

  \item Protons have a charge of $e^+$ and a spin of $\frac{1}{2}$, neutrons have no charge, but the same spin value

  \item Neutrons were discovered in 1932

  \item The atomic number is the sum of protons and neutrons

    \begin{itemize}

      \item Thus, a nucleus with a mass number $A$ contains $Z$ protons and $N=A-Z$ neutrons

      \item Neutrons and protons are referred to as nucleons

      \item Nuclei with similar $Z$ may have different $N$

      \item For example, fully specifying a hydrogen atom, we may get:

        \begin{itemize}

          \item $^1_1H_0$

          \item $^2_1H_1$

          \item $^3_1H_2$

        \end{itemize}

      \item These are known as isotopes

    \end{itemize}

    \section{Nuclear Size and Shape}

  \item It is observed that the density of a nucleus does not depend on its atomic number $A$

    $$\dfrac{N+Z}{\frac{4}{3}\pi R^2}\Rightarrow \frac{A}{R^2}\quad\text{ is constant}$$

  \item The Nucleus Radius

    \begin{itemize}

      \item We know $R=R_oA^{\frac{1}{3}}$

      \item We also know $\rho=\frac{m}{V}$

      \item Combining these, we get

    \end{itemize}

  \item Nuclear Binding Energy

    \begin{itemize}

      \item $E_{b}=\left[ Nm_n+Zm_p-m\left( ^A_ZZ_N \right) \right]$

      \item $\frac{E_b}{A}=\frac{\text{Binding energy}}{\text{\# of molecules}}$

      \item The binding energy to remove the least bound nucleon from the nucleus is on the order of $[\si{\mega\eV}]$, while the ionization energy of an electron is on the order of $[\si{\eV}]$

    \end{itemize}

  \item Stable and Unstable Nuclei

    \begin{itemize}

      \item Most nuclei are not stable

      \item They decay to lighter, more stable ones

      \item Decay Processes:

        \begin{itemize}

          \item $\alpha-$decay — Emission of a helium nucleus, $^4_2He_2$

          \item $\beta-$decay — Involves the ejection of a beta particle $\rightarrow$ $^A_ZX\rightarrow ^A_{Z+1}X' + e^- + \beta$ or $^A_ZX\rightarrow ^A_{Z-1}X' + e^+ + \beta$

        \end{itemize}

      \item Activity and Decay Probabilities

        \begin{itemize}

          \item This is the rate at which $N$ unstable nuclei decay

          \item Number of decays per second

          \item Units of Curies are used ($1[\si{\Curie}]=3.7\cdot10^{10}[\text{decays}/\si{\second}]$)

          \item $P(t)$ is the probability of decay after a given time $t$

          \item Decay probability per nucleus per second is called the decay constant, $\lambda$

          \item $\lambda$ is decay process/element dependent

          \item The activity, $a$, would be $a=\lambda N$

        \end{itemize}

      \item Exponential Law of Radioactive Decay:

        \begin{itemize}

          \item $a=-\frac{dN}{dt}\Rightarrow \lambda N=-\frac{dN}{dt}$

          \item Solving this yields $N=N_oe^{-\lambda t}$

          \item This means $a=a_oe^{-\lambda t}$

        \end{itemize}

    \end{itemize}

  \item Particles

    \begin{itemize}

      \item Spin 1/2, tiny mass, no charge: neutrinos

      \item An anti-neutrino is indicated by $n\rightarrow p$

    \end{itemize}

\end{itemize}

\end{document}

