%%%%%%%%%%%%%%%%%%%%%%%%%%%%%%%%%%%%%%%%%%%%%%%%%%%%%%%%%%%%%%%%%%%%%%%%%%%%%%%%%%%%%%%%%%%%%%%%%%%%%%%%%%%%%%%%%%%%%%%%%%%%%%%%%%%%%%%%%%%%%%%%%%%%%%%%%%%%%%%%%%%
% Written By Michael Brodskiy
% Class: Modern Physics
% Professor: Q. Yan
%%%%%%%%%%%%%%%%%%%%%%%%%%%%%%%%%%%%%%%%%%%%%%%%%%%%%%%%%%%%%%%%%%%%%%%%%%%%%%%%%%%%%%%%%%%%%%%%%%%%%%%%%%%%%%%%%%%%%%%%%%%%%%%%%%%%%%%%%%%%%%%%%%%%%%%%%%%%%%%%%%%

\include{Includes.tex}

\title{Nuclear Physics}
\date{\today}
\author{Michael Brodskiy\\ \small Professor: Q. Yan}

\begin{document}

\maketitle

\newpage

\tableofcontents

\newpage

\begin{itemize}

    \section{Nuclear Structure}

  \item The size of an atom is $1[\si{\angstrom}]$

  \item The size of the nucleus is $.001[\si{\angstrom}]$, or $1[\si{\femto\meter}]$

  \item The repulsive positive charges in the nucleus are held together by the strong nuclear force

    \begin{itemize}

      \item Has a very short range

    \end{itemize}

  \item Protons have a charge of $e^+$ and a spin of $\frac{1}{2}$, neutrons have no charge, but the same spin value

  \item Neutrons were discovered in 1932

  \item The atomic number is the sum of protons and neutrons

    \begin{itemize}

      \item Thus, a nucleus with a mass number $A$ contains $Z$ protons and $N=A-Z$ neutrons

      \item Neutrons and protons are referred to as nucleons

      \item Nuclei with similar $Z$ may have different $N$

      \item For example, fully specifying a hydrogen atom, we may get:

        \begin{itemize}

          \item $^1_1H_0$

          \item $^2_1H_1$

          \item $^3_1H_2$

        \end{itemize}

      \item These are known as isotopes

    \end{itemize}

    \section{Nuclear Size and Shape}

  \item It is observed that the density of a nucleus does not depend on its atomic number $A$

    $$\dfrac{N+Z}{\frac{4}{3}\pi R^2}\Rightarrow \frac{A}{R^2}\quad\text{ is constant}$$

  \item The Nucleus Radius

    \begin{itemize}

      \item We know $R=R_oA^{\frac{1}{3}}$

      \item We also know $\rho=\frac{m}{V}$

      \item Combining these, we get

    \end{itemize}

  \item Nuclear Binding Energy

    \begin{itemize}

      \item $E_{b}=\left[ Nm_n+Zm_p-m\left( ^A_ZZ_N \right) \right]$

      \item $\frac{E_b}{A}=\frac{\text{Binding energy}}{\text{\# of molecules}}$

      \item The binding energy to remove the least bound nucleon from the nucleus is on the order of $[\si{\mega\eV}]$, while the ionization energy of an electron is on the order of $[\si{\eV}]$

    \end{itemize}

  \item Stable and Unstable Nuclei

    \begin{itemize}

      \item Most nuclei are not stable

      \item They decay to lighter, more stable ones

      \item Decay Processes:

        \begin{itemize}

          \item $\alpha-$decay — Emission of a helium nucleus, $^4_2He$

        \end{itemize}

    \end{itemize}

\end{itemize}

\end{document}

