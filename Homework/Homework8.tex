%%%%%%%%%%%%%%%%%%%%%%%%%%%%%%%%%%%%%%%%%%%%%%%%%%%%%%%%%%%%%%%%%%%%%%%%%%%%%%%%%%%%%%%%%%%%%%%%%%%%%%%%%%%%%%%%%%%%%%%%%%%%%%%%%%%%%%%%%%%%%%%%%%%%%%%%%%%%%%%%%%%
% Written By Michael Brodskiy
% Class: Modern Physics
% Professor: Q. Yan
%%%%%%%%%%%%%%%%%%%%%%%%%%%%%%%%%%%%%%%%%%%%%%%%%%%%%%%%%%%%%%%%%%%%%%%%%%%%%%%%%%%%%%%%%%%%%%%%%%%%%%%%%%%%%%%%%%%%%%%%%%%%%%%%%%%%%%%%%%%%%%%%%%%%%%%%%%%%%%%%%%%

\documentclass[12pt]{article} 
\usepackage{alphalph}
\usepackage[utf8]{inputenc}
\usepackage[russian,english]{babel}
\usepackage{titling}
\usepackage{amsmath}
\usepackage{graphicx}
\usepackage{enumitem}
\usepackage{amssymb}
\usepackage[super]{nth}
\usepackage{everysel}
\usepackage{ragged2e}
\usepackage{geometry}
\usepackage{multicol}
\usepackage{fancyhdr}
\usepackage{cancel}
\usepackage{siunitx}
\usepackage{physics}
\usepackage{tikz}
\usepackage{mathdots}
\usepackage{yhmath}
\usepackage{cancel}
\usepackage{color}
\usepackage{array}
\usepackage{multirow}
\usepackage{gensymb}
\usepackage{tabularx}
\usepackage{extarrows}
\usepackage{booktabs}
\usepackage{expl3}
\usepackage[version=4]{mhchem}
\usepackage{hpstatement}
\usetikzlibrary{fadings}
\usetikzlibrary{patterns}
\usetikzlibrary{shadows.blur}
\usetikzlibrary{shapes}

\geometry{top=1.0in,bottom=1.0in,left=1.0in,right=1.0in}
\newcommand{\subtitle}[1]{%
  \posttitle{%
    \par\end{center}
    \begin{center}\large#1\end{center}
    \vskip0.5em}%

}
\usepackage{hyperref}
\hypersetup{
colorlinks=true,
linkcolor=blue,
filecolor=magenta,      
urlcolor=blue,
citecolor=blue,
}


\title{Homework 8}
\date{April 18, 2023}
\author{Michael Brodskiy\\ \small Professor: Q. Yan}

\begin{document}

\maketitle

\newpage

\begin{enumerate}

    \section*{A One-Dimensional Atom}

  \item The probability of finding an electron in a range may be found using $P=\displaystyle\int_a^b|\psi(x)|^2\,dx$

    $$P=\int_0^{a_o} |\psi(x)|^2\,dx$$
    $$\psi(x)=2x\left( \frac{1}{a_o} \right)^{\frac{3}{2}}e^{-\frac{x}{a_o}}$$
    $$\int_0^{a_o} \frac{4x^2}{a_o^3}e^{-\frac{2x}{a_o}}\, dx$$
    $$\frac{4}{a_o^3}\int_0^{a_o} x^2e^{-\frac{2x}{a_o}}\,dx$$

    Using a mathematical solver, we get:

    $$\frac{4}{a_o^3}\int_0^{a_o} x^2e^{-\frac{2x}{a_o}}\,dx=.323$$

    There is a 32.3\% probability the electron is in this range

    \section*{Hydrogen Atom Wave Functions}

  \item The possible quantum numbers are:

    \begin{center}
      \begin{tabular}[h!]{|c|c|c|c|}
        \hline
        $n$ & $l$ & $m_l$ & $m_s$\\
        \hline
        $4$ & $0$ & $0$ & $\pm1/2$\\
        \hline
        $4$ & $1$ & $-1$ & $\pm1/2$\\
        \hline
        $4$ & $1$ & $0$ & $\pm1/2$\\
        \hline
        $4$ & $1$ & $1$ & $\pm1/2$\\
        \hline
        $4$ & $2$ & $-2$ & $\pm1/2$\\
        \hline
        $4$ & $2$ & $-1$ & $\pm1/2$\\
        \hline
        $4$ & $2$ & $0$ & $\pm1/2$\\
        \hline
        $4$ & $2$ & $1$ & $\pm1/2$\\
        \hline
        $4$ & $2$ & $2$ & $\pm1/2$\\
        \hline
        $4$ & $3$ & $-3$ & $\pm1/2$\\
        \hline
        $4$ & $3$ & $-2$ & $\pm1/2$\\
        \hline
        $4$ & $3$ & $-1$ & $\pm1/2$\\
        \hline
        $4$ & $3$ & $0$ & $\pm1/2$\\
        \hline
        $4$ & $3$ & $1$ & $\pm1/2$\\
        \hline
        $4$ & $3$ & $2$ & $\pm1/2$\\
        \hline
        $4$ & $3$ & $3$ & $\pm1/2$\\
        \hline
      \end{tabular}
    \end{center}

    \section*{Hydrogen Atom Wave Functions 2}

  \item We know the spherical wave equation is:

    $$-\frac{\hbar^2}{2m}\left[ \frac{\partial^2}{\partial r^2}+\frac{2}{r}\frac{\partial \Psi}{\partial r}+\frac{1}{r^2\sin(\theta)}\frac{\partial}{\partial\theta}\left( \sin(\theta)\frac{\partial\Psi}{\partial\theta} \right)+\frac{1}{r^2\sin^2(\theta)}\frac{\partial^2}{\partial\phi^2} \right]+U(r)\Psi(r,\theta.\phi)=E\Psi(r,\theta,\phi)$$

    We also know that the ground state wave function of a hydrogen atom is:

    $$\psi=\frac{1}{\sqrt{\pi}}\left( \frac{1}{a_o} \right)^{\frac{3}{2}}e^{-\frac{r}{a_o}}$$
    $$\frac{2}{r}\frac{\partial \Psi}{\partial r}=-\frac{2}{r\sqrt{\pi}}\left( \frac{1}{a_o} \right)^{\frac{5}{2}}e^{-\frac{r}{a_o}}$$
    $$\frac{\partial\Psi^2}{\partial r^2}=\frac{1}{\sqrt{\pi}}\left( \frac{1}{a_o} \right)^{\frac{7}{2}}e^{-\frac{r}{a_o}}$$

    Plugging this into the differential equation, we get:

    $$-\frac{\hbar^2}{2m}\left[ \frac{1}{\sqrt{\pi}}\left( \frac{1}{a_o} \right)^{\frac{7}{2}}e^{-\frac{r}{a_o}}-\frac{2}{r\sqrt{\pi}}\left( \frac{1}{a_o} \right)^{\frac{5}{2}}e^{-\frac{r}{a_o}} \right]+U(r)\Psi=E\Psi$$

    This can be rewritten as:

    $$-\frac{\hbar^2}{2m}\left[ \left( \frac{1}{a_o} \right)^2\cancel{\Psi}-\frac{2}{ra_o}\cancel{\Psi} \right]+U(r)\cancel{\Psi}=E\cancel{\Psi}$$

    We know $E$ is independent of $r$, so we get:

    $$E=-\frac{\hbar^2}{2m}\left( \frac{1}{a_o} \right)^2$$

    Because we know that $a_o=\frac{4\pi\varepsilon_o\hbar^2}{me^2}$, we can plug this in:

    $$E=-\frac{\hbar^2}{2m}\frac{m^2e^4}{\hbar^4}=-\frac{me^4}{32\pi^2\varepsilon_o^2\hbar^2}$$

    This means $E_o=-13.6[\si{\eV}]$; thus, this is a valid solution.

    \section*{Radical Probability Densities}

  \item The radial wave function of the state where $n=2$ and $l=0$ is:

    $$\psi=\left( 2-\frac{r}{a_o} \right)\left( \frac{1}{a_o} \right)^{\frac{3}{2}}e^{-\dfrac{r}{2a_o}}$$

    The probability density is given by:

  $$P_{2,0}=4\pi r^2|\psi|^2$$

  Differentiating this and setting it equal to 0 will yield the highest probability value:

  $$\dfrac{d}{dr}\left( 4\pi r^2|\psi|^2 \right)=0$$

  This becomes:

  $$\frac{d}{dr}\left( 4\pi r^2\left( 2-\frac{r}{a_o} \right)^2\left( \frac{1}{a_o} \right)^3  e^{-\frac{r}{a_o}}\right)$$

  Simplifying, this turns into:

  $$\frac{d}{dr}\left( \left( \frac{16\pi r^2}{a_o^3}-\frac{16\pi r^3}{a_o^4}+\frac{4\pi r^4}{a_o^5} \right)e^{-\frac{r}{a_o}} \right)$$

  Differentiating, this becomes:

  $$\left( \frac{32\pi r}{a_o^3}-\frac{48\pi r^2}{a_o^4}+\frac{16\pi r^3}{a_o^5} \right)e^{-\frac{r}{a_o}}-\frac{1}{a_o}\left( \frac{16\pi r^2}{a_o^3}-\frac{16\pi r^3}{a_o^4}+\frac{4\pi r^4}{a_o^5} \right)e^{-\frac{r}{a_o}}=0$$

  The exponential terms cancel out, which leaves us with:

  $$\left( \frac{32\pi r}{a_o^3}-\frac{64\pi r^2}{a_o^4}+\frac{32\pi r^3}{a_o^5}-\frac{4\pi r^4}{a_o^6} \right)=0$$

  Simplifying further:

  $$8-\frac{16r}{a_o}+\frac{8r^2}{a_o^2}-\frac{r^3}{a_o^3}=0$$

  Using a numerical solver, the roots of this are found to be:

  $$r=2a_o,(3\pm\sqrt{5})a_o$$

  Peaks occur at $r=(3\pm\sqrt{5})a_o$

    \section*{Intrinsic Spin}

  \item

    \begin{enumerate}

      \item The degeneracy may be calculated using the formula $2n^2$; for the $n=5$ energy level, it is found that there are $2n^2=2(5)^2=50$ degeneracies

      \item The possible combinations for $n=5$ are as follows:

        $$n=5\right l = \left\{\begin{array}{c c} 0, & m_l =0\\1, & m_l=\left\{\begin{array}{c} 0\\ \pm1 \end{array}\\2, & m_l=\left\{\begin{array}{c} 0\\\pm1\\\pm2\end{array}\\3, & m_l=\left\{\begin{array}{c} 0\\\pm1\\\pm2\\\pm3\end{array}\\4, & m_l=\left\{\begin{array}{c} 0\\\pm1\\\pm2\\\pm3\\\pm4\end{array}\end{array}$$

    \end{enumerate}

    Counting the possible values, there are 25. When including spin in the calculation, this doubles the degeneracy values, as, for any quantum number, spin may be $\pm\frac{1}{2}$. As such, there are $2(25)=50$ degeneracies

\end{enumerate}

\end{document}

