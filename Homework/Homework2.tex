%%%%%%%%%%%%%%%%%%%%%%%%%%%%%%%%%%%%%%%%%%%%%%%%%%%%%%%%%%%%%%%%%%%%%%%%%%%%%%%%%%%%%%%%%%%%%%%%%%%%%%%%%%%%%%%%%%%%%%%%%%%%%%%%%%%%%%%%%%%%%%%%%%%%%%%%%%%%%%%%%%%
% Written By Michael Brodskiy
% Class: Modern Physics
% Professor: Q. Yan
%%%%%%%%%%%%%%%%%%%%%%%%%%%%%%%%%%%%%%%%%%%%%%%%%%%%%%%%%%%%%%%%%%%%%%%%%%%%%%%%%%%%%%%%%%%%%%%%%%%%%%%%%%%%%%%%%%%%%%%%%%%%%%%%%%%%%%%%%%%%%%%%%%%%%%%%%%%%%%%%%%%

\documentclass[12pt]{article} 
\usepackage{alphalph}
\usepackage[utf8]{inputenc}
\usepackage[russian,english]{babel}
\usepackage{titling}
\usepackage{amsmath}
\usepackage{graphicx}
\usepackage{enumitem}
\usepackage{amssymb}
\usepackage[super]{nth}
\usepackage{everysel}
\usepackage{ragged2e}
\usepackage{geometry}
\usepackage{multicol}
\usepackage{fancyhdr}
\usepackage{cancel}
\usepackage{siunitx}
\usepackage{physics}
\usepackage{tikz}
\usepackage{mathdots}
\usepackage{yhmath}
\usepackage{cancel}
\usepackage{color}
\usepackage{array}
\usepackage{multirow}
\usepackage{gensymb}
\usepackage{tabularx}
\usepackage{extarrows}
\usepackage{booktabs}
\usepackage{expl3}
\usepackage[version=4]{mhchem}
\usepackage{hpstatement}
\usetikzlibrary{fadings}
\usetikzlibrary{patterns}
\usetikzlibrary{shadows.blur}
\usetikzlibrary{shapes}

\geometry{top=1.0in,bottom=1.0in,left=1.0in,right=1.0in}
\newcommand{\subtitle}[1]{%
  \posttitle{%
    \par\end{center}
    \begin{center}\large#1\end{center}
    \vskip0.5em}%

}
\usepackage{hyperref}
\hypersetup{
colorlinks=true,
linkcolor=blue,
filecolor=magenta,      
urlcolor=blue,
citecolor=blue,
}


\title{Homework 2}
\date{January 29, 2023}
\author{Michael Brodskiy\\ \small Professor: Q. Yan}

\begin{document}

\maketitle

\newpage

\begin{enumerate}

    \section{Twins}

  \item

    \begin{enumerate}

      \item 

        \begin{itemize}

          \item Total distance for Mary-Kate: $ 2L_0 = 32[\text{light-years}]$
            
          \item Total time for Mary-Kate: $\Delta t_0 = 20[\text{yr}]$


            $$v\Delta t_0 = 2L$$
            $$v\Delta t_0=2L_0\sqrt{1-\dfrac{v^2}{c^2}}$$
            $$v=\frac{2L_0\sqrt{1-\dfrac{v^2}{c^2}}}{\Delta t_0}$$
            $$v^2=\frac{(32)^2c^2\left(1-\dfrac{v^2}{c^2}\right)}{20^2}$$
            $$v^2=2.56c^2-2.56v^2$$
            $$3.56v^2=2.56c^2$$
            $$\boxed{v=.848c}$$

        \end{itemize}

      \item According to the result from (a), the speed at which Mary-Kate traveled is $.848c$. Applying the time dilation formula using this knowledge yields:

        $$\Delta t = \frac{20}{\sqrt{1-(.848)^2}}$$
        $$\Delta t = 37.7[\text{yr}]$$

        So Ashley is $37.7-20=17.7$ years older than Mary-Kate when Mary-Kate returns

    \end{enumerate}

    \section{Spherical Waves}

  \item The Lorentz Transformation is given as:

      $$\left\{\begin{array}{l}x'=\dfrac{x-ut}{\sqrt{1-\frac{u^2}{c^2}}}\\y'=y\\z'=z\\t'=\dfrac{t-\frac{u}{c^2}x}{\sqrt{1-\frac{u^2}{c^2}}}\\ \end{array}$$

        Because it is stated that the pulse begins at time $t=0$, it can be assumed that, at this time, the pulse is not moving. As such, the transformations reduce to:

        $$\left\{\begin{array}{l}x'=\dfrac{x-(0)t}{\sqrt{1-\frac{0^2}{c^2}}}=x\\y'=y\\z'=z\\t'=\dfrac{t-\frac{0}{c^2}x}{\sqrt{1-\frac{0^2}{c^2}}}=t\\ \end{array}$$

          In this manner, we substitute each transformation into the original formula, which yields:

          $$\boxed{x'^2+y'^2+z'^2=(ct')^2=0}$$

    \section{Pole Vaulting}

  \item

    \begin{enumerate}

      \item Using the length contraction formula, the proper length ($L_0$) as $20[\si{\meter}]$, and the observed length $(L)$ as $10[\si{\meter}]$, we obtain:

        $$10 = 20\sqrt{1-\dfrac{u^2}{c^2}}$$
        $$.25=1-\dfrac{u^2}{c^2}$$
        $$.25c^2=c^2-u^2$$
        $$u^2=.75c^2$$
        $$\boxed{u=.866c}$$

        As Ming's speed in reference frame of observer $O$

      \item As with the phenomenon of clock desynchronization, because Ming is moving towards the garage, it appears to him that the farther door closes first. In such a manner, the time difference between the state alternation of the doors (closed to open) can be expressed as the following:

        $$\Delta t'=\dfrac{\frac{uL}{c^2}}{\sqrt{1-\frac{u^2}{c^2}}}$$
        $$\Delta t'= \frac{\frac{(10)(.866)}{c}}{\sqrt{1-(.866)^2}}$$
        $$\Delta t'= 5.7328\cdot10^{-8}[\si{\second}]$$

        In this time, Ming is able to travel the following distance:

        $$(5.7328)(3)(.866)=14.894[\si{\meter}]$$

        Because the garage is actually 10 meters long, and Ming travels $14.894[\si{\meter}]$, he is able to enter and exit the garage prior to it closing. In this manner, he is $4.894[\si{\meter}]$ away from the garage by the time the door closes. As such, Ming is able to safely enter and exit the 10$[\si{\meter}]$ garage despite having a 20$[\si{\meter}]$ pole.

    \end{enumerate}

    \section{Meson Decay}

  \item $\pi$ Meson speed: $v_x=\pm.815c$; $K$ Meson speed: $u=.453c$

    This would mean, using the Lorentz velocity transformation, the first $\pi$ meson particle would have a speed of:

    $$v_{x1}'=\frac{.453-.815}{1-(.815)(.453)}c$$
    $$\boxed{v_{x1}'=-.574c}$$

    And the other, using $v_x=-.815c$, would have a speed of:

    $$v_{x2}'=\frac{.453+.815}{1+(.815)(.453)}c$$
    $$\boxed{v_{x2}'=.926c}$$

    \section{Meter Stick}

  \item Because the motion is parallel, only the $x$ component of the meter stick experiences length contraction. Thus, to find the components, we would perform the following:

    $$L_y=(1[\si{\meter}])\sin(30)=.5[\si{\meter}]$$
    $$L_x=(1[\si{\meter}])\cos(30)=.866[\si{\meter}]$$

      Following contraction, $L_x$ becomes:

      $$L_x'=.866(\sqrt{1-.81})=.378[\si{\meter}]$$

      Thus, the new length of the meter stick becomes:

      $$L=\sqrt{(.378)^2+(.5)^2}$$
      $$\boxed{L=.627[\si{\meter}]}$$


\end{enumerate}

\end{document}

