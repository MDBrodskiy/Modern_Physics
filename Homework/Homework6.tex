%%%%%%%%%%%%%%%%%%%%%%%%%%%%%%%%%%%%%%%%%%%%%%%%%%%%%%%%%%%%%%%%%%%%%%%%%%%%%%%%%%%%%%%%%%%%%%%%%%%%%%%%%%%%%%%%%%%%%%%%%%%%%%%%%%%%%%%%%%%%%%%%%%%%%%%%%%%%%%%%%%%
% Written By Michael Brodskiy
% Class: Modern Physics
% Professor: Q. Yan
%%%%%%%%%%%%%%%%%%%%%%%%%%%%%%%%%%%%%%%%%%%%%%%%%%%%%%%%%%%%%%%%%%%%%%%%%%%%%%%%%%%%%%%%%%%%%%%%%%%%%%%%%%%%%%%%%%%%%%%%%%%%%%%%%%%%%%%%%%%%%%%%%%%%%%%%%%%%%%%%%%%

\documentclass[12pt]{article} 
\usepackage{alphalph}
\usepackage[utf8]{inputenc}
\usepackage[russian,english]{babel}
\usepackage{titling}
\usepackage{amsmath}
\usepackage{graphicx}
\usepackage{enumitem}
\usepackage{amssymb}
\usepackage[super]{nth}
\usepackage{everysel}
\usepackage{ragged2e}
\usepackage{geometry}
\usepackage{multicol}
\usepackage{fancyhdr}
\usepackage{cancel}
\usepackage{siunitx}
\usepackage{physics}
\usepackage{tikz}
\usepackage{mathdots}
\usepackage{yhmath}
\usepackage{cancel}
\usepackage{color}
\usepackage{array}
\usepackage{multirow}
\usepackage{gensymb}
\usepackage{tabularx}
\usepackage{extarrows}
\usepackage{booktabs}
\usepackage{expl3}
\usepackage[version=4]{mhchem}
\usepackage{hpstatement}
\usetikzlibrary{fadings}
\usetikzlibrary{patterns}
\usetikzlibrary{shadows.blur}
\usetikzlibrary{shapes}

\geometry{top=1.0in,bottom=1.0in,left=1.0in,right=1.0in}
\newcommand{\subtitle}[1]{%
  \posttitle{%
    \par\end{center}
    \begin{center}\large#1\end{center}
    \vskip0.5em}%

}
\usepackage{hyperref}
\hypersetup{
colorlinks=true,
linkcolor=blue,
filecolor=magenta,      
urlcolor=blue,
citecolor=blue,
}


\title{Homework 6}
\date{March 29, 2023}
\author{Michael Brodskiy\\ \small Professor: Q. Yan}

\begin{document}

\maketitle

\newpage

\begin{enumerate}

    \section*{Permitted Wave Functions}

    \begin{enumerate}

      \item One reason why this function is not permitted is because it violates the normalization condition, $\displaystyle \int_{-\infty}^{\infty}|\psi(x)|^2\,dx=1$; more specifically, solving for the boundary conditions makes it violate this:

        $$A\cos(kx)=B\sin(kx)$$
        $$A\cos(0)=B\sin(0)$$
        $$A=0$$

        Differentiating to find $B$:

        $$0=Bk\cos(kx)$$
        $$B=0$$

        Because both constants are zero, the integral over the entire boundary does not equal 1.

      \item $\psi(x)=\dfrac{Ae^{-kx}}{x}$ can not be a solution because it is discontinuous; at the point $x=0$, the function has a discontinuity.

      \item $A\sin^{-1}(kx)$ can not be a solution because it is discontinuous. $\sin^{-1}$ is only valid for values in the range $\left(-\dfrac{\pi}{2}, \dfrac{\pi}{2}\right)$, and, thus, it must have a discontinuity somewhere in its domain, unless $k$ were to have the value of zero; in such a case, the function would violate the normalization condition, as it would be zero over its whole domain.

      \item $A\tan(kx)$ can not be a solution because it is discontinuous every $n\pi$ intervals.

    \end{enumerate}

    \section*{The Schr\"odinger Equation}

    For a particle with $\psi(x)=Cxe^{-bx}$, plugging into the Schr\"odinger equation would yield:

    $$\left( -\dfrac{\hbar^2}{2m} \right)(b^2Cxe^{-bx}-2bCe^{-bx})+U(x)(Cxe^{-bx})=E(Cxe^{-bx})$$
    $$E=\left( -\dfrac{\hbar^2}{2m} \right)\left(b^2-\frac{2b}{x}\right)+U(x)$$

    The $x$ terms balance and cancel out because $E$ is constant:

    $$E=-\dfrac{\hbar^2b^2}{2m}$$

    This makes $U(x)$ equal to the final term left when $E$ cancels out:

    $$U(x)=-\dfrac{b\hbar^2}{mx}$$

    \section*{Expectation Values}

    \begin{enumerate}

      \item In ground state ($n=1$)

        $$\psi(x)=\sqrt{\dfrac{2}{L}}sin\left( \dfrac{n\pi x}{L} \right)$$
        $$\int_0^L \dfrac{2}{L}\sin^2\left( \dfrac{\pi x}{L} \right) x\,dx=\frac{L}{2}$$

      \item In first excited state ($n=2$)

        $$\int_0^L \dfrac{2}{L}\sin^2\left( \dfrac{2\pi x}{L} \right) x\,dx=\frac{L}{2}$$

        It appears that the expected position value of any such particle in a unidimensional well would be $\frac{L}{2}$

    \end{enumerate}

    \newpage

    \section*{A Particle in a 3D Box}

    \begin{figure}[h!]
      \centering
      \tikzset{every picture/.style={line width=0.75pt}} %set default line width to 0.75pt        

\begin{tikzpicture}[x=0.75pt,y=0.75pt,yscale=-1,xscale=1]
%uncomment if require: \path (0,538); %set diagram left start at 0, and has height of 538

%Shape: Rectangle [id:dp7431820249349796] 
\draw  [color={rgb, 255:red, 255; green, 255; blue, 255 }  ,draw opacity=1 ][fill={rgb, 255:red, 255; green, 255; blue, 255 }  ,fill opacity=1 ] (253,87) -- (362,87) -- (362,127) -- (253,127) -- cycle ;
%Shape: Rectangle [id:dp4619815785817558] 
\draw  [color={rgb, 255:red, 255; green, 255; blue, 255 }  ,draw opacity=1 ][fill={rgb, 255:red, 255; green, 255; blue, 255 }  ,fill opacity=1 ] (253,127) -- (362,127) -- (362,167) -- (253,167) -- cycle ;
%Shape: Rectangle [id:dp3359362966762425] 
\draw  [color={rgb, 255:red, 255; green, 255; blue, 255 }  ,draw opacity=1 ][fill={rgb, 255:red, 255; green, 255; blue, 255 }  ,fill opacity=1 ] (253,167) -- (362,167) -- (362,207) -- (253,207) -- cycle ;
%Shape: Rectangle [id:dp5119516390611945] 
\draw  [color={rgb, 255:red, 255; green, 255; blue, 255 }  ,draw opacity=1 ][fill={rgb, 255:red, 255; green, 255; blue, 255 }  ,fill opacity=1 ] (253,207) -- (362,207) -- (362,247) -- (253,247) -- cycle ;
%Shape: Rectangle [id:dp00577409558737596] 
\draw  [color={rgb, 255:red, 255; green, 255; blue, 255 }  ,draw opacity=1 ][fill={rgb, 255:red, 255; green, 255; blue, 255 }  ,fill opacity=1 ] (253,247) -- (362,247) -- (362,287) -- (253,287) -- cycle ;
%Shape: Rectangle [id:dp9392024809985304] 
\draw  [color={rgb, 255:red, 255; green, 255; blue, 255 }  ,draw opacity=1 ][fill={rgb, 255:red, 255; green, 255; blue, 255 }  ,fill opacity=1 ] (253,287) -- (362,287) -- (362,327) -- (253,327) -- cycle ;
%Shape: Rectangle [id:dp36507090012453736] 
\draw  [color={rgb, 255:red, 255; green, 255; blue, 255 }  ,draw opacity=1 ][fill={rgb, 255:red, 255; green, 255; blue, 255 }  ,fill opacity=1 ] (253,327) -- (362,327) -- (362,367) -- (253,367) -- cycle ;
%Shape: Rectangle [id:dp3845863388878359] 
\draw  [color={rgb, 255:red, 255; green, 255; blue, 255 }  ,draw opacity=1 ][fill={rgb, 255:red, 255; green, 255; blue, 255 }  ,fill opacity=1 ] (253,367) -- (362,367) -- (362,407) -- (253,407) -- cycle ;
%Shape: Rectangle [id:dp08289501558521084] 
\draw  [color={rgb, 255:red, 255; green, 255; blue, 255 }  ,draw opacity=1 ][fill={rgb, 255:red, 255; green, 255; blue, 255 }  ,fill opacity=1 ] (253,407) -- (362,407) -- (362,447) -- (253,447) -- cycle ;
%Shape: Rectangle [id:dp30336721777460673] 
\draw  [color={rgb, 255:red, 255; green, 255; blue, 255 }  ,draw opacity=1 ][fill={rgb, 255:red, 255; green, 255; blue, 255 }  ,fill opacity=1 ] (253,47) -- (362,47) -- (362,87) -- (253,87) -- cycle ;
%Shape: Rectangle [id:dp33732499056647836] 
\draw   (99,407) -- (253,407) -- (253,447) -- (99,447) -- cycle ;
%Shape: Rectangle [id:dp11504885678961019] 
\draw   (99,367) -- (253,367) -- (253,407) -- (99,407) -- cycle ;
%Shape: Rectangle [id:dp469041150289242] 
\draw   (99,327) -- (253,327) -- (253,367) -- (99,367) -- cycle ;
%Shape: Rectangle [id:dp5969911762901989] 
\draw   (99,287) -- (253,287) -- (253,327) -- (99,327) -- cycle ;
%Shape: Rectangle [id:dp6587152869266266] 
\draw   (99,247) -- (253,247) -- (253,287) -- (99,287) -- cycle ;
%Shape: Rectangle [id:dp7605249930787139] 
\draw   (99,207) -- (253,207) -- (253,247) -- (99,247) -- cycle ;
%Shape: Rectangle [id:dp411629440002667] 
\draw   (99,167) -- (253,167) -- (253,207) -- (99,207) -- cycle ;
%Shape: Rectangle [id:dp8033789873500379] 
\draw   (99,127) -- (253,127) -- (253,167) -- (99,167) -- cycle ;
%Shape: Rectangle [id:dp8981685875637919] 
\draw   (99,47) -- (253,47) -- (253,87) -- (99,87) -- cycle ;
%Shape: Rectangle [id:dp9045254208550346] 
\draw   (99,87) -- (253,87) -- (253,127) -- (99,127) -- cycle ;
%Shape: Rectangle [id:dp2632550671115941] 
\draw  [color={rgb, 255:red, 255; green, 255; blue, 255 }  ,draw opacity=1 ][fill={rgb, 255:red, 255; green, 255; blue, 255 }  ,fill opacity=1 ] (253,7) -- (362,7) -- (362,47) -- (253,47) -- cycle ;
%Shape: Rectangle [id:dp6780964060092816] 
\draw  [color={rgb, 255:red, 255; green, 255; blue, 255 }  ,draw opacity=1 ][fill={rgb, 255:red, 255; green, 255; blue, 255 }  ,fill opacity=1 ] (362,7) -- (471,7) -- (471,47) -- (362,47) -- cycle ;
%Shape: Rectangle [id:dp016786318374576004] 
\draw  [color={rgb, 255:red, 255; green, 255; blue, 255 }  ,draw opacity=1 ][fill={rgb, 255:red, 255; green, 255; blue, 255 }  ,fill opacity=1 ] (362,87) -- (471,87) -- (471,127) -- (362,127) -- cycle ;
%Shape: Rectangle [id:dp8106205400386881] 
\draw  [color={rgb, 255:red, 255; green, 255; blue, 255 }  ,draw opacity=1 ][fill={rgb, 255:red, 255; green, 255; blue, 255 }  ,fill opacity=1 ] (362,127) -- (471,127) -- (471,167) -- (362,167) -- cycle ;
%Shape: Rectangle [id:dp27924210424820184] 
\draw  [color={rgb, 255:red, 255; green, 255; blue, 255 }  ,draw opacity=1 ][fill={rgb, 255:red, 255; green, 255; blue, 255 }  ,fill opacity=1 ] (362,167) -- (471,167) -- (471,207) -- (362,207) -- cycle ;
%Shape: Rectangle [id:dp3175983250673733] 
\draw  [color={rgb, 255:red, 255; green, 255; blue, 255 }  ,draw opacity=1 ][fill={rgb, 255:red, 255; green, 255; blue, 255 }  ,fill opacity=1 ] (362,207) -- (471,207) -- (471,247) -- (362,247) -- cycle ;
%Shape: Rectangle [id:dp05894387939144652] 
\draw  [color={rgb, 255:red, 255; green, 255; blue, 255 }  ,draw opacity=1 ][fill={rgb, 255:red, 255; green, 255; blue, 255 }  ,fill opacity=1 ] (362,247) -- (471,247) -- (471,287) -- (362,287) -- cycle ;
%Shape: Rectangle [id:dp6291481253412834] 
\draw  [color={rgb, 255:red, 255; green, 255; blue, 255 }  ,draw opacity=1 ][fill={rgb, 255:red, 255; green, 255; blue, 255 }  ,fill opacity=1 ] (362,287) -- (471,287) -- (471,327) -- (362,327) -- cycle ;
%Shape: Rectangle [id:dp1413064646582567] 
\draw  [color={rgb, 255:red, 255; green, 255; blue, 255 }  ,draw opacity=1 ][fill={rgb, 255:red, 255; green, 255; blue, 255 }  ,fill opacity=1 ] (362,327) -- (471,327) -- (471,367) -- (362,367) -- cycle ;
%Shape: Rectangle [id:dp19360669921120954] 
\draw  [color={rgb, 255:red, 255; green, 255; blue, 255 }  ,draw opacity=1 ][fill={rgb, 255:red, 255; green, 255; blue, 255 }  ,fill opacity=1 ] (362,367) -- (471,367) -- (471,407) -- (362,407) -- cycle ;
%Shape: Rectangle [id:dp5127588021437042] 
\draw  [color={rgb, 255:red, 255; green, 255; blue, 255 }  ,draw opacity=1 ][fill={rgb, 255:red, 255; green, 255; blue, 255 }  ,fill opacity=1 ] (362,407) -- (471,407) -- (471,447) -- (362,447) -- cycle ;
%Shape: Rectangle [id:dp1536824207359908] 
\draw  [color={rgb, 255:red, 255; green, 255; blue, 255 }  ,draw opacity=1 ][fill={rgb, 255:red, 255; green, 255; blue, 255 }  ,fill opacity=1 ] (362,47) -- (471,47) -- (471,87) -- (362,87) -- cycle ;
%Shape: Rectangle [id:dp20780962546904314] 
\draw  [color={rgb, 255:red, 255; green, 255; blue, 255 }  ,draw opacity=1 ][fill={rgb, 255:red, 255; green, 255; blue, 255 }  ,fill opacity=1 ] (362,447) -- (471,447) -- (471,487) -- (362,487) -- cycle ;
%Shape: Rectangle [id:dp19273902809704002] 
\draw  [color={rgb, 255:red, 255; green, 255; blue, 255 }  ,draw opacity=1 ][fill={rgb, 255:red, 255; green, 255; blue, 255 }  ,fill opacity=1 ] (253,447) -- (362,447) -- (362,487) -- (253,487) -- cycle ;
%Shape: Rectangle [id:dp9594313478215819] 
\draw   (99,447) -- (253,447) -- (253,487) -- (99,487) -- cycle ;

% Text Node
\draw (307.5,27) node    {$n( x,y,z)$};
% Text Node
\draw (307.5,467) node  [font=\footnotesize]  {$( 1,1,1)$};
% Text Node
\draw (416.5,27) node   [align=left] {Degeneracies};
% Text Node
\draw (416.5,467) node  [font=\footnotesize] [align=left] {None};
% Text Node
\draw (307.5,427) node  [font=\footnotesize]  {$( 2,\ 1,\ 1)$};
% Text Node
\draw (307.5,387) node  [font=\footnotesize]  {$( 2,2,1)$};
% Text Node
\draw (307.5,347) node  [font=\footnotesize]  {$( 3,1,1)$};
% Text Node
\draw (307.5,307) node  [font=\footnotesize]  {$( 2,2,2)$};
% Text Node
\draw (307.5,267) node  [font=\footnotesize]  {$( 3,2,1)$};
% Text Node
\draw (307.5,227) node  [font=\footnotesize]  {$( 3,2,2)$};
% Text Node
\draw (307.5,187) node  [font=\footnotesize]  {$( 4,1,1)$};
% Text Node
\draw (307.5,147) node  [font=\footnotesize]  {$( 3,3,1)$};
% Text Node
\draw (307.5,107) node  [font=\footnotesize]  {$( 4,2,1)$};
% Text Node
\draw (307.5,67) node  [font=\footnotesize]  {$( 3,3,2)$};
% Text Node
\draw (416.5,427) node  [font=\scriptsize]  {$( 1,2,1) ,\ ( 1,\ 1,\ 2)$};
% Text Node
\draw (176,467) node    {$3E_{o}$};
% Text Node
\draw (176,427) node    {$6E_{o}$};
% Text Node
\draw (176,387) node    {$9E_{o}$};
% Text Node
\draw (176,347) node    {$11E_{o}$};
% Text Node
\draw (176,307) node    {$12E_{o}$};
% Text Node
\draw (176,267) node    {$14E_{o}$};
% Text Node
\draw (176,227) node    {$17E_{o}$};
% Text Node
\draw (176,187) node    {$18E_{o}$};
% Text Node
\draw (176,147) node    {$19E_{o}$};
% Text Node
\draw (176,107) node    {$21E_{o}$};
% Text Node
\draw (176,67) node    {$22E_{o}$};
% Text Node
\draw (416.5,387) node  [font=\scriptsize]  {$( 1,2,2) ,\ ( 2,\ 1,\ 2)$};
% Text Node
\draw (416.5,347) node  [font=\scriptsize]  {$( 1,3,1) ,\ ( 1,1,3)$};
% Text Node
\draw (416.5,307) node  [font=\footnotesize] [align=left] {None};
% Text Node
\draw (416.5,267) node  [font=\scriptsize]  {$ \begin{array}{l}
( 1,2,3) ,\ ( 1,3,2) ,\ ( 3,1,2) ,\\
( 2,1,3) ,\ ( 2,3,1)
\end{array}$};
% Text Node
\draw (416.5,227) node  [font=\scriptsize]  {$( 2,3,2) ,\ ( 2,2,3)$};
% Text Node
\draw (416.5,187) node  [font=\scriptsize]  {$( 1,4,1) ,\ ( 1,1,4)$};
% Text Node
\draw (416.5,147) node  [font=\scriptsize]  {$( 3,1,3) ,\ ( 1,3,3)$};
% Text Node
\draw (416.5,107) node  [font=\scriptsize]  {$ \begin{array}{l}
( 1,2,4) ,\ ( 1,4,2) ,\ ( 4,1,2) ,\\
( 2,1,4) ,\ ( 2,4,1)
\end{array}$};
% Text Node
\draw (416.5,67) node  [font=\scriptsize]  {$( 3,2,3) ,\ ( 2,3,3)$};


\end{tikzpicture}

      \caption{Energy Levels of $E_n(n_x,n_y,n_z)=E_o(n_x^2+n_y^2+n_z^2)$}
      \label{fig:1}
    \end{figure}

    \section*{Quantum Simple Harmonic Oscillator}

    \begin{enumerate}

      \item 

      \item 

    \end{enumerate}

\end{enumerate}

\end{document}

