%%%%%%%%%%%%%%%%%%%%%%%%%%%%%%%%%%%%%%%%%%%%%%%%%%%%%%%%%%%%%%%%%%%%%%%%%%%%%%%%%%%%%%%%%%%%%%%%%%%%%%%%%%%%%%%%%%%%%%%%%%%%%%%%%%%%%%%%%%%%%%%%%%%%%%%%%%%%%%%%%%%
% Written By Michael Brodskiy
% Class: Modern Physics
% Professor: Q. Yan
%%%%%%%%%%%%%%%%%%%%%%%%%%%%%%%%%%%%%%%%%%%%%%%%%%%%%%%%%%%%%%%%%%%%%%%%%%%%%%%%%%%%%%%%%%%%%%%%%%%%%%%%%%%%%%%%%%%%%%%%%%%%%%%%%%%%%%%%%%%%%%%%%%%%%%%%%%%%%%%%%%%

\include{Includes.tex}

\title{Homework 1}
\date{January 22, 2023}
\author{Michael Brodskiy\\ \small Professor: Q. Yan}

\begin{document}

\maketitle

\newpage

\begin{enumerate}

  \item Binomial Expansion Exercise $\rightarrow (1+b)^n=1+nb+\frac{n(n-1)}{2!}b^2$

    \begin{enumerate}

      \item $\left(\sqrt{1-\dfrac{u^2}{c^2}}\right)^{-1}=\left( 1-\dfrac{u^2}{c^2} \right)^{-\frac{1}{2}}\longrightarrow b=-\dfrac{u^2}{c^2}; n=-\dfrac{1}{2}\longrightarrow\boxed{\gamma=1+\dfrac{u^2}{2c^2}+\dfrac{3u^4}{8c^4}}$

      \item $\sqrt{1-\dfrac{u^2}{c^2}}=\left( 1-\dfrac{u^2}{c^2} \right)^{\frac{1}{2}}\longrightarrow b=-\dfrac{u^2}{c^2}; n=\dfrac{1}{2}\longrightarrow\boxed{\dfrac{1}{\gamma}=1-\dfrac{u^2}{2c^2}-\dfrac{u^4}{8c^4}}$
        
    \end{enumerate}

  \item Inertial Reference Frames

    \begin{enumerate}

      \item The two are not necessarily equal; the results depend on the motion of each respective frame of reference, mostly due to the effect of length contraction

      \item Again, this value would not necessarily be the same in each frame; here, the mass of the proton is dependent on the relative speed of the proton with respect to each frame

      \item Einstein's second postulate means that this is equal in both reference frames

      \item This would not necessarily be equal in different frames, as, due to the effect of time dilation, if movement with respect to one reference frame is much closer to the speed of light than the other, the faster moving object will experience time in a slower manner

      \item Because Newton's first law depends on relative motion of a frame of reference, results would be different depending on which frame this is experienced in

      \item This would be the same, as the order of periodic elements is not influenced by motion, and, thus would remain the same regardless of reference frames

      \item The charge of an electron is a fundamental constant, and most definitely unaffected by motion, and, thus, it would remain the same in both frames

    \end{enumerate}
    
  \item Michelson-Morley Experiment

    \begin{enumerate}

      \item 

        \begin{itemize}

          \item Horizontal distance to travel: $L$ to the right and $L$ to the left; Horizontal velocity: $c-v$ to the right, and $c+v$ to the left. Because time is distance over velocity, we obtain:

            $$t_{\parallel}=\frac{L}{c-v}+\frac{L}{c+v}=\frac{2Lc}{c^2-v^2}=\frac{2L}{c}\frac{1}{1-\frac{v^2}{c^2}}$$

          \item Vertical distance to travel $L$ up and $L$ down; Velocity vector: $c$, horizontal velocity $u\longrightarrow$ vertical velocity : $\sqrt{c^2-v^2}$

            $$t_{\perp}=\frac{2L}{\sqrt{c^2-v^2}}=\frac{2L}{c}\frac{1}{\sqrt{1-\frac{v^2}{c^2}}}$$

            $$\boxed{\Delta t=\frac{2L}{c}\left( \frac{1}{1-\frac{v^2}{c^2}} - \frac{1}{\sqrt{1-\frac{v^2}{c^2}}} \right)}$$

        \end{itemize}

      \item

            $$f=\frac{v}{\lambda}; f=\frac{N}{\Delta t}; N=\frac{v}{\lambda}\Delta t$$
            $$\Delta t=\frac{4L}{c}\left( \frac{1}{1-\frac{v^2}{c^2}} - \frac{1}{\sqrt{1-\frac{v^2}{c^2}}} \right)$$
            $$\Delta t = \frac{4 (11)}{30000}\left( \frac{1}{1-\frac{(30000)^2}{(3\cdot10^8)^2}}-\frac{1}{\sqrt{1-\frac{(30000)^2}{(3\cdot10^8)^2}}} \right)$$
            $$=7.333[\si{\pico\second}]$$
            $$(7.333\cdot10^{-12}) \cdot \frac{30000}{500\cdot10^{-9}}=(7.333)\frac{30}{500} = .44$$

            \begin{itemize}

              \item $\boxed{\text{There was a fringe shift of .44}}$

            \end{itemize}

    \end{enumerate}

\end{enumerate}

\end{document}

