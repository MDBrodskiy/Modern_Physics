%%%%%%%%%%%%%%%%%%%%%%%%%%%%%%%%%%%%%%%%%%%%%%%%%%%%%%%%%%%%%%%%%%%%%%%%%%%%%%%%%%%%%%%%%%%%%%%%%%%%%%%%%%%%%%%%%%%%%%%%%%%%%%%%%%%%%%%%%%%%%%%%%%%%%%%%%%%%%%%%%%%
% Written By Michael Brodskiy
% Class: Modern Physics
% Professor: Q. Yan
%%%%%%%%%%%%%%%%%%%%%%%%%%%%%%%%%%%%%%%%%%%%%%%%%%%%%%%%%%%%%%%%%%%%%%%%%%%%%%%%%%%%%%%%%%%%%%%%%%%%%%%%%%%%%%%%%%%%%%%%%%%%%%%%%%%%%%%%%%%%%%%%%%%%%%%%%%%%%%%%%%%

\include{Includes.tex}

\usepackage[dvipsnames,table]{xcolor}
\usepackage{siunitx} % SI-units
\usepackage{pgfplots}
\usepgfplotslibrary{units} % to add units easily to axis
\usepgfplotslibrary{fillbetween} % to fill inbetween curves
\usepgfplotslibrary{colormaps} % to create colormaps

\title{Homework 4}
\date{February 23, 2023}
\author{Michael Brodskiy\\ \small Professor: Q. Yan}

\begin{document}

\maketitle

\newpage

\begin{enumerate}

    \section*{Light Bulb and Photons}

    \begin{enumerate}

      \item The energy of the photons may be calculated using $E=hf$

        $$E=\dfrac{hc}{\lambda}=\dfrac{\left(6.626\cdot10^{-34}\right)\left(3\cdot10^8\right)}{550\cdot10^{-9}}=3.614\cdot10^{-19}[\si{\joule}]$$
        \begin{itemize}

          \item The total amount of energy in 55 watts can be found as:

            $$\dfrac{55}{E}=15.27\cdot10^{19}[\text{photons per second}]$$

          \item With 75\% efficiency, this becomes

            $$.75\cdot15.27\cdot10^{19}=11.45\cdot10^{19}[\text{photons per second}]$$

          \item Converting to hours, we finally get:

            $$11.45\cdot10^{19}\cdot3600=4.122\cdot10^{23}[\text{photons per hour}]$$

        \end{itemize}

      \item

        \begin{itemize}

          \item The area of the plate is:

            $$(.01)^2=.0001[\si{\meter\squared}]$$

          \item The area of the emitted light within the radius of the plate is:

            $$4\pi r^2=4\pi(1)^2=12.566[\si{\meter\squared}]$$

          \item Using area proportions and the number calculated in (a), we get:

            $$\left(11.45\cdot10^{19}\right)\dfrac{.0001}{12.566}=9.11\cdot10^{14}[\text{photons}]$$

        \end{itemize}

    \end{enumerate}

    \section*{Compton Scattering}

    \begin{itemize}

      \item The formula for energy difference of a scattered electron is:

        $$\dfrac{1}{E'}-\dfrac{1}{E}=\dfrac{1}{m_ec^2}(1-\cos(\theta))$$

      \item The maximum kinetic energy given to the electron occurs when the subsequent energy of the photon is minimal, or when $\theta = 180$. Thus, we obtain:

        $$\dfrac{1}{E'}=\dfrac{2}{m_ec^2}+\dfrac{1}{E}$$
        $$E'=\left(\dfrac{2}{m_ec^2}+\dfrac{1}{E}\right)^{-1}$$

    \end{itemize}

    \section*{Thermal Radiation}

    \begin{enumerate}

      \item 

        \begin{itemize}

          \item Using Wien's displacement:

            $$\lambda_{max}=\dfrac{2.9\cdot10^{-3}}{T}$$
            $$\lambda_{max}=\dfrac{2.9\cdot10^{-3}}{273+34}$$
            $$\lambda_{max}=9.45[\si{\micro\meter}]$$

          \item Per the EM spectrum below, this is in the infrared range

            \begin{figure}[h!]
              \centering
              \include{Figures/EMSpec}
              \caption{Electromagnetic Spectrum}
              \label{fig:1}
            \end{figure}

        \end{itemize}

      \item

        \begin{itemize}

          \item Assuming a human body to be roughly $2[\si{\meter\squared}]$ in surface area, Stefan's law may be used:

            $$P=\sigma AT^4$$
            $$\left( 5.67\cdot10^{-8} \right)(2)(273+34)^4=1007.3[\si{\watt}]$$

        \end{itemize}

    \end{enumerate}

    \section*{Photoelectric Effect}

    \begin{itemize}

      \item The formula relation for voltage and work function is:

        $$e^-V=hf-\phi$$

      \item With copper, this means:

        $$V=\dfrac{\dfrac{hc}{\lambda}-\phi_1}{e^-}$$

      \item With sodium, $V$ would be:

        $$V_s=\dfrac{\dfrac{hc}{\lambda}-\phi_2}{e^-}$$

      \item This means:

        $$V_s=V - \dfrac{\phi_2-\phi_1}{e^-}$$

    \end{itemize}

\end{enumerate}

\end{document}

