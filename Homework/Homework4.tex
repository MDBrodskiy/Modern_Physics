%%%%%%%%%%%%%%%%%%%%%%%%%%%%%%%%%%%%%%%%%%%%%%%%%%%%%%%%%%%%%%%%%%%%%%%%%%%%%%%%%%%%%%%%%%%%%%%%%%%%%%%%%%%%%%%%%%%%%%%%%%%%%%%%%%%%%%%%%%%%%%%%%%%%%%%%%%%%%%%%%%%
% Written By Michael Brodskiy
% Class: Modern Physics
% Professor: Q. Yan
%%%%%%%%%%%%%%%%%%%%%%%%%%%%%%%%%%%%%%%%%%%%%%%%%%%%%%%%%%%%%%%%%%%%%%%%%%%%%%%%%%%%%%%%%%%%%%%%%%%%%%%%%%%%%%%%%%%%%%%%%%%%%%%%%%%%%%%%%%%%%%%%%%%%%%%%%%%%%%%%%%%

\documentclass[12pt]{article} 
\usepackage{alphalph}
\usepackage[utf8]{inputenc}
\usepackage[russian,english]{babel}
\usepackage{titling}
\usepackage{amsmath}
\usepackage{graphicx}
\usepackage{enumitem}
\usepackage{amssymb}
\usepackage[super]{nth}
\usepackage{everysel}
\usepackage{ragged2e}
\usepackage{geometry}
\usepackage{multicol}
\usepackage{fancyhdr}
\usepackage{cancel}
\usepackage{siunitx}
\usepackage{physics}
\usepackage{tikz}
\usepackage{mathdots}
\usepackage{yhmath}
\usepackage{cancel}
\usepackage{color}
\usepackage{array}
\usepackage{multirow}
\usepackage{gensymb}
\usepackage{tabularx}
\usepackage{extarrows}
\usepackage{booktabs}
\usepackage{expl3}
\usepackage[version=4]{mhchem}
\usepackage{hpstatement}
\usetikzlibrary{fadings}
\usetikzlibrary{patterns}
\usetikzlibrary{shadows.blur}
\usetikzlibrary{shapes}

\geometry{top=1.0in,bottom=1.0in,left=1.0in,right=1.0in}
\newcommand{\subtitle}[1]{%
  \posttitle{%
    \par\end{center}
    \begin{center}\large#1\end{center}
    \vskip0.5em}%

}
\usepackage{hyperref}
\hypersetup{
colorlinks=true,
linkcolor=blue,
filecolor=magenta,      
urlcolor=blue,
citecolor=blue,
}


\usepackage[dvipsnames,table]{xcolor}
\usepackage{siunitx} % SI-units
\usepackage{pgfplots}
\usepgfplotslibrary{units} % to add units easily to axis
\usepgfplotslibrary{fillbetween} % to fill inbetween curves
\usepgfplotslibrary{colormaps} % to create colormaps

\title{Homework 4}
\date{February 23, 2023}
\author{Michael Brodskiy\\ \small Professor: Q. Yan}

\begin{document}

\maketitle

\newpage

\begin{enumerate}

    \section*{Light Bulb and Photons}

    \begin{enumerate}

      \item The energy of the photons may be calculated using $E=hf$

        $$E=\dfrac{hc}{\lambda}=\dfrac{\left(6.626\cdot10^{-34}\right)\left(3\cdot10^8\right)}{550\cdot10^{-9}}=3.614\cdot10^{-19}[\si{\joule}]$$
        \begin{itemize}

          \item The total amount of energy in 55 watts can be found as:

            $$\dfrac{55}{E}=15.27\cdot10^{19}[\text{photons per second}]$$

          \item With 75\% efficiency, this becomes

            $$.75\cdot15.27\cdot10^{19}=11.45\cdot10^{19}[\text{photons per second}]$$

          \item Converting to hours, we finally get:

            $$11.45\cdot10^{19}\cdot3600=4.122\cdot10^{23}[\text{photons per hour}]$$

        \end{itemize}

      \item

        \begin{itemize}

          \item The area of the plate is:

            $$(.01)^2=.0001[\si{\meter\squared}]$$

          \item The area of the emitted light within the radius of the plate is:

            $$4\pi r^2=4\pi(1)^2=12.566[\si{\meter\squared}]$$

          \item Using area proportions and the number calculated in (a), we get:

            $$\left(11.45\cdot10^{19}\right)\dfrac{.0001}{12.566}=9.11\cdot10^{14}[\text{photons}]$$

        \end{itemize}

    \end{enumerate}

    \section*{Compton Scattering}

    \begin{itemize}

      \item The formula for energy difference of a scattered electron is:

        $$\dfrac{1}{E'}-\dfrac{1}{E}=\dfrac{1}{m_ec^2}(1-\cos(\theta))$$

      \item The maximum kinetic energy given to the electron occurs when the subsequent energy of the photon is minimal, or when $\theta = 180$. Thus, we obtain:

        $$\dfrac{1}{E'}=\dfrac{2}{m_ec^2}+\dfrac{1}{E}$$
        $$E'=\left(\dfrac{2}{m_ec^2}+\dfrac{1}{E}\right)^{-1}$$

    \end{itemize}

    \section*{Thermal Radiation}

    \begin{enumerate}

      \item 

        \begin{itemize}

          \item Using Wien's displacement:

            $$\lambda_{max}=\dfrac{2.9\cdot10^{-3}}{T}$$
            $$\lambda_{max}=\dfrac{2.9\cdot10^{-3}}{273+34}$$
            $$\lambda_{max}=9.45[\si{\micro\meter}]$$

          \item Per the EM spectrum below, this is in the infrared range

            \begin{figure}[h!]
              \centering
              
\pgfplotsset{width=12.2cm, height=7cm}
\pgfplotsset{compat=newest} %(making it only compatalbe with
%new releases of pgfplots)
\pgfdeclarehorizontalshading{visiblelight}{50bp}{
color(0.00000000000000bp)=(violet);
color(8.33333333333333bp)=(blue);
color(16.66666666666670bp)=(cyan);
color(25.00000000000000bp)=(green);
color(33.33333333333330bp)=(yellow); color(41.66666666666670bp)=(orange);
color(50.00000000000000bp)=(red)
}%

\begin{tikzpicture}[fill between/on layer={axis grid}]
\begin{axis}[
xlabel={Wavelength},
xticklabel style = {font=\tiny,yshift=0.2ex},
xmin=10^-5,
xmax=10^9,
x unit=\si{\micro\meter},
xmode=log,
ymin=0,
ymax=1,
height=3cm,
yticklabels={},
ytick=\empty,
legend cell align=left,
legend style={at={(0.85,-0.77)},anchor=north}
]
\addplot[draw=none, name path=start, forget plot] coordinates{(10^-5,0)(10^-5,1)};
\addplot[draw=none, name path=gamma, forget plot] coordinates{(10^-3,0)(10^-3,1)};
\addplot[draw=none, name path=xrays, forget plot] coordinates{(10^-2,0)(10^-2,1)};
\addplot[draw=none, name path=uv, forget plot] coordinates{(0.4,0)(0.4,1)};
\addplot[draw=none, name path=visible, forget plot] coordinates{(0.7,0)(0.7,1)};
\addplot[draw=none, name path=ir, forget plot] coordinates{(10^2.5,0)(10^2.5,1)};
\addplot[draw=none, name path=microwave, forget plot] coordinates{(10^5,0)(10^5,1)};
\addplot[draw=none, name path=radiowave, forget plot] coordinates{(10^9,0)(10^9,1)};
\addplot[violet!20, area legend] fill between[of=start and gamma];
\addlegendentry{$\gamma$-ray}
\addplot[violet!60, area legend] fill between[of=gamma and xrays];
\addlegendentry{X-ray}
\addplot[violet, area legend] fill between[of=xrays and uv];
\addlegendentry{Ultra violet}
\addplot[shading=visiblelight, area legend] fill between[of=uv and visible];
\addlegendentry{Visible light}
\addplot[red!50, area legend] fill between[of=visible and ir];
\addlegendentry{Infrared}
\addplot[red, area legend] fill between[of=ir and microwave];
\addlegendentry{Micro wave}
\addplot[Brown, area legend] fill between[of=microwave and radiowave];
\addlegendentry{Radio wave}
\end{axis}
\end{tikzpicture}

              \caption{Electromagnetic Spectrum}
              \label{fig:1}
            \end{figure}

        \end{itemize}

      \item

        \begin{itemize}

          \item Assuming a human body to be roughly $2[\si{\meter\squared}]$ in surface area, Stefan's law may be used:

            $$P=\sigma AT^4$$
            $$\left( 5.67\cdot10^{-8} \right)(2)(273+34)^4=1007.3[\si{\watt}]$$

        \end{itemize}

    \end{enumerate}

    \section*{Photoelectric Effect}

    \begin{itemize}

      \item The formula relation for voltage and work function is:

        $$e^-V=hf-\phi$$

      \item With copper, this means:

        $$V=\dfrac{\dfrac{hc}{\lambda}-\phi_1}{e^-}$$

      \item With sodium, $V$ would be:

        $$V_s=\dfrac{\dfrac{hc}{\lambda}-\phi_2}{e^-}$$

      \item This means:

        $$V_s=V - \dfrac{\phi_2-\phi_1}{e^-}$$

    \end{itemize}

\end{enumerate}

\end{document}

