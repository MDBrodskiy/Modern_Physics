%%%%%%%%%%%%%%%%%%%%%%%%%%%%%%%%%%%%%%%%%%%%%%%%%%%%%%%%%%%%%%%%%%%%%%%%%%%%%%%%%%%%%%%%%%%%%%%%%%%%%%%%%%%%%%%%%%%%%%%%%%%%%%%%%%%%%%%%%%%%%%%%%%%%%%%%%%%%%%%%%%%
% Written By Michael Brodskiy
% Class: Modern Physics
% Professor: Q. Yan
%%%%%%%%%%%%%%%%%%%%%%%%%%%%%%%%%%%%%%%%%%%%%%%%%%%%%%%%%%%%%%%%%%%%%%%%%%%%%%%%%%%%%%%%%%%%%%%%%%%%%%%%%%%%%%%%%%%%%%%%%%%%%%%%%%%%%%%%%%%%%%%%%%%%%%%%%%%%%%%%%%%

\include{Includes.tex}

\usepackage[dvipsnames,table]{xcolor}
\usepackage{siunitx} % SI-units
\usepackage{pgfplots}
\usepgfplotslibrary{units} % to add units easily to axis
\usepgfplotslibrary{fillbetween} % to fill inbetween curves
\usepgfplotslibrary{colormaps} % to create colormaps

\title{Homework 5}
\date{March 7, 2023}
\author{Michael Brodskiy\\ \small Professor: Q. Yan}

\begin{document}

\maketitle

\newpage

\begin{enumerate}

  \item 
    
    \section*{Estimations of de Broglie Waves}

    \begin{enumerate}

      \item Boltzmann Constant in $\dfrac{\si{\joule}}{\si{\kelvin}}=1.381\cdot10^{-23}$; Room temperature in $\si{\kelvin}=293$

        $$K_{avg}=\frac{3}{2}k_bT$$
        $$p=\sqrt{2mK_{avg}}$$
        $$\lambda_{avg}=\frac{6.626\cdot10^{-34}}{\sqrt{3\cdot28.013\cdot1.66\cdot10^{-27}\cdot293.15\cdot1.38\cdot10^{-24}}}$$
        $$\lambda_{avg}=2.79\cdot10^{-11}[\si{\meter}]=.0279[\si{\nano\meter}]$$

      \item $.02[\si{\eV}]=3.204\cdot10^{-21}[\si{\joule}]$

        $$\lambda_{avg}=\frac{h}{\sqrt{2mK}}$$
        $$\frac{h}{\sqrt{2mK}}=\frac{6.626\cdot10^{-34}}{\sqrt{2\cdot1.675\cdot10^{-27}\cdot3.204\cdot10^{-21}}}$$
        $$\lambda_{avg}=2.022\cdot10^{-10}[\si{\meter}]=.2022[\si{\nano\meter}]$$

      \item $1\left[ \frac{\si{\meter}}{\text{yr}} \right]=3.17\cdot10^{-8}\left[ \frac{\si{\meter}}{\si{\second}} \right]$

        $$\lambda_{avg}=\frac{6.626\cdot10^{-34}}{.001\cdot3.17\cdot10^{-8}}$$
        $$\lambda_{avg}=2.09\cdot10^{-23}[\si{\meter}]=20.9[\si{\yocto\meter}]$$

    \end{enumerate}

  \item

    \section*{de Broglie Wave of a Proton}

    \begin{enumerate}

      \item $L=.01[\si{\meter}]$, so the round-trip distance for one oscillation is $2L=.02[\si{\meter}]$. Thus:

        $$2L=n\lambda$$

        Rearranging, we get:

        $$\lambda=\frac{2L}{n}$$

      \item 

    \end{enumerate}

  \item

    \section*{$e^-$ and $e^+$ Annihilation}

    \begin{enumerate}

      \item 

        $$\lambda_{e^-,e^+}=\frac{6.626\cdot10^{-34}}{9.109\cdot10^{-31}\cdot3\cdot10^{6}}=.2425[\si{\nano\meter}]$$

      \item 

        $$E=mc^2+\sum K$$

        $$\left( 9.109\cdot10^{-31} \right)\left( 3\cdot10^{8} \right)^2+\frac{1}{2}\left( 9.109\cdot10^{-31} \right)\left( 3\cdot10^6 \right)^2$$
        $$E=8.2\cdot10^{-14}[\si{\joule}]=511,803.7[\si{\eV}]$$

        $$\lambda=\frac{E}{hc}=\frac{511,803.7}{1240}=412.75[\si{\nano\meter}]$$

        $$E=pc\footnote{for photons}$$
        $$p=\frac{E}{c}=2.73\cdot10^{-22}\left[ \frac{\si{\kilo\gram\meter}}{\si{\second}} \right]=.511\left[ \frac{\si{\mega\eV}}{\text{c}} \right]$$

    \end{enumerate}

  \item

    \section*{Uncertainty}

    \begin{enumerate}

      \item 

      \item 

    \end{enumerate}

\end{enumerate}

\end{document}

