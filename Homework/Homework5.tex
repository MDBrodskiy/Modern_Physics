%%%%%%%%%%%%%%%%%%%%%%%%%%%%%%%%%%%%%%%%%%%%%%%%%%%%%%%%%%%%%%%%%%%%%%%%%%%%%%%%%%%%%%%%%%%%%%%%%%%%%%%%%%%%%%%%%%%%%%%%%%%%%%%%%%%%%%%%%%%%%%%%%%%%%%%%%%%%%%%%%%%
% Written By Michael Brodskiy
% Class: Modern Physics
% Professor: Q. Yan
%%%%%%%%%%%%%%%%%%%%%%%%%%%%%%%%%%%%%%%%%%%%%%%%%%%%%%%%%%%%%%%%%%%%%%%%%%%%%%%%%%%%%%%%%%%%%%%%%%%%%%%%%%%%%%%%%%%%%%%%%%%%%%%%%%%%%%%%%%%%%%%%%%%%%%%%%%%%%%%%%%%

\documentclass[12pt]{article} 
\usepackage{alphalph}
\usepackage[utf8]{inputenc}
\usepackage[russian,english]{babel}
\usepackage{titling}
\usepackage{amsmath}
\usepackage{graphicx}
\usepackage{enumitem}
\usepackage{amssymb}
\usepackage[super]{nth}
\usepackage{everysel}
\usepackage{ragged2e}
\usepackage{geometry}
\usepackage{multicol}
\usepackage{fancyhdr}
\usepackage{cancel}
\usepackage{siunitx}
\usepackage{physics}
\usepackage{tikz}
\usepackage{mathdots}
\usepackage{yhmath}
\usepackage{cancel}
\usepackage{color}
\usepackage{array}
\usepackage{multirow}
\usepackage{gensymb}
\usepackage{tabularx}
\usepackage{extarrows}
\usepackage{booktabs}
\usepackage{expl3}
\usepackage[version=4]{mhchem}
\usepackage{hpstatement}
\usetikzlibrary{fadings}
\usetikzlibrary{patterns}
\usetikzlibrary{shadows.blur}
\usetikzlibrary{shapes}

\geometry{top=1.0in,bottom=1.0in,left=1.0in,right=1.0in}
\newcommand{\subtitle}[1]{%
  \posttitle{%
    \par\end{center}
    \begin{center}\large#1\end{center}
    \vskip0.5em}%

}
\usepackage{hyperref}
\hypersetup{
colorlinks=true,
linkcolor=blue,
filecolor=magenta,      
urlcolor=blue,
citecolor=blue,
}


\usepackage[dvipsnames,table]{xcolor}
\usepackage{siunitx} % SI-units
\usepackage{pgfplots}
\usepgfplotslibrary{units} % to add units easily to axis
\usepgfplotslibrary{fillbetween} % to fill inbetween curves
\usepgfplotslibrary{colormaps} % to create colormaps

\title{Homework 5}
\date{March 7, 2023}
\author{Michael Brodskiy\\ \small Professor: Q. Yan}

\begin{document}

\maketitle

\newpage

\begin{enumerate}

  \item 
    
    \section*{Estimations of de Broglie Waves}

    \begin{enumerate}

      \item Boltzmann Constant in $\dfrac{\si{\joule}}{\si{\kelvin}}=1.381\cdot10^{-23}$; Room temperature in $\si{\kelvin}=293$

        $$K_{avg}=\frac{3}{2}k_bT$$
        $$p=\sqrt{2mK_{avg}}$$
        $$\lambda_{avg}=\frac{6.626\cdot10^{-34}}{\sqrt{3\cdot28.013\cdot1.66\cdot10^{-27}\cdot293.15\cdot1.38\cdot10^{-24}}}$$
        $$\boxed{\lambda_{avg}=2.79\cdot10^{-11}[\si{\meter}]=.0279[\si{\nano\meter}]}$$

      \item $.02[\si{\eV}]=3.204\cdot10^{-21}[\si{\joule}]$

        $$\lambda_{avg}=\frac{h}{\sqrt{2mK}}$$
        $$\frac{h}{\sqrt{2mK}}=\frac{6.626\cdot10^{-34}}{\sqrt{2\cdot1.675\cdot10^{-27}\cdot3.204\cdot10^{-21}}}$$
        $$\boxed{\lambda_{avg}=2.022\cdot10^{-10}[\si{\meter}]=.2022[\si{\nano\meter}]}$$

      \item $1\left[ \frac{\si{\meter}}{\text{yr}} \right]=3.17\cdot10^{-8}\left[ \frac{\si{\meter}}{\si{\second}} \right]$

        $$\lambda_{avg}=\frac{6.626\cdot10^{-34}}{.001\cdot3.17\cdot10^{-8}}$$
        $$\boxed{\lambda_{avg}=2.09\cdot10^{-23}[\si{\meter}]=20.9[\si{\yocto\meter}]}$$

    \end{enumerate}

  \item

    \section*{de Broglie Wave of a Proton}

    \begin{enumerate}

      \item $L=.01[\si{\meter}]$, so the round-trip distance for one oscillation is $2L=.02[\si{\meter}]$. Thus:

        $$2L=n\lambda$$

        Rearranging, we get:

        $$\boxed{\lambda=\frac{2L}{n}}$$

      \item 

        $$\lambda=\frac{h}{\sqrt{2mK}}$$
        $$\frac{h}{\sqrt{2mK}}=\frac{2L}{n}$$
        $$2mK=\left(\dfrac{nh}{2L}\right)^2$$
        $$K=\dfrac{\dfrac{n^2h^2}{4L^2}}{2m}$$
        

        \begin{center}
          For $n=1$
        \end{center}

        $$\boxed{K_1=\dfrac{(6.626\cdot10^{-34})^2}{\dfrac{4(.01)^2}{2(1.67\cdot10^{-27})}}=2.05\cdot10^{-18}[\si{\eV}]}$$

        \begin{center}
          For $n=2$
        \end{center}

        $$\boxed{K_2=\dfrac{4(6.626\cdot10^{-34})^2}{\dfrac{4(.01)^2}{2(1.67\cdot10^{-27})}}=8.2\cdot10^{-18}[\si{\eV}]}$$

    \end{enumerate}

  \item

    \section*{$e^-$ and $e^+$ Annihilation}

    \begin{enumerate}

      \item 

        $$\boxed{\lambda_{e^-,e^+}=\frac{6.626\cdot10^{-34}}{9.109\cdot10^{-31}\cdot3\cdot10^{6}}=.2425[\si{\nano\meter}]}$$

      \item 

        $$E=mc^2+\sum K$$

        $$\left( 9.109\cdot10^{-31} \right)\left( 3\cdot10^{8} \right)^2+\frac{1}{2}\left( 9.109\cdot10^{-31} \right)\left( 3\cdot10^6 \right)^2$$
        $$\boxed{E=8.2\cdot10^{-14}[\si{\joule}]=511,803.7[\si{\eV}]}$$

        $$\boxed{\lambda=\frac{E}{hc}=\frac{511,803.7}{1240}=412.75[\si{\nano\meter}]}$$

        $$E=pc\footnote{for photons}$$
        $$\boxed{p=\frac{E}{c}=2.73\cdot10^{-22}\left[ \frac{\si{\kilo\gram\meter}}{\si{\second}} \right]=.511\left[ \frac{\si{\mega\eV}}{\text{c}} \right]}$$

    \end{enumerate}

  \item

    \section*{Uncertainty}

    \begin{enumerate}

      \item 

        $$\Delta x\Delta p \approx \frac{h}{2\pi}$$
        $$\lambda\Delta p \approx \frac{h}{2\pi}$$
        $$\Delta p \approx \left(\frac{h}{\lambda}\right)\frac{1}{2\pi}$$
        $$\boxed{\Delta p \approx \frac{p}{2\pi}}$$

      \item 

        $$\lambda=\frac{h}{\gamma p}$$
        $$E_k=m_nc^2(\gamma-1)$$
        $$\frac{E_k}{m_nc^2}=\gamma-1$$
        $$\gamma=\frac{10}{(6.231\cdot10^{12})\cdot1.675\cdot10^{-27}\cdot\left(3\cdot10^8\right)^2}+1$$
        $$\gamma=1.0107$$
        $$v=\sqrt{c^2-c^2\left( \frac{1}{1.0107} \right)^2}$$
        $$v=.145c$$
        $$\lambda=\frac{6.626\cdot10^{-34}}{1.0107\cdot1.675\cdot10^{-27}\cdot.145c}$$
        $$\boxed{\lambda=9\cdot10^{-15}[\si{\meter}]=9[\si{\femto\meter}]}$$

        $9[\si{\femto\meter}]$ is greater than $1[\si{\femto\meter}]$ but less than $10[\si{\femto\meter}]$, so atom nuclei may be used to demonstrate the wave nature of $10[\si{\mega\eV}]$ neutrons.

    \end{enumerate}

\end{enumerate}

\end{document}

